\chapter{Software a altres sistemes operatius}
\label{cap:software-other}

Tal i com s'ha dit al principi d'aquest treball de final de grau, l'objectiu
del projecte és realitzar un programari que pugui funcionar amb la gran majoria
de sistemes operatius disponibles al mercat. Tanmateix, s'ha vist que les
llibreries per a modificar aspectes gràfics dels dispositius varien molt entre
cada \est{kernel}, pel que la possibilitat de realitzar un únic programa que
funcioni per a tots els sistemes operatius queda descartada.

Degut a les limitacions de temps i dedicació d'aquest treball, no es podrà
repetir totes les tasques realitzades en el capítol anterior amb la resta
de sistemes operatius. Tanmateix, es pretén aprendre el funcionament d'aquests
per a deixar un camí més senzill si mai es vulgués acabar implementant un
programari per a aquests equipaments.

En aquest capítol s'explorarà la possibilitat de llegir dades d'un sensor
\acro{usb} i modificar l'orientació d'una pantalla en els dos sistemes operatius
que, juntament amb Linux, s'utilitzen en gairebé tots els ordinadors personals.
No es fixa com a objectiu crear un programa insta\l.lable per a cada un d'ells
degut a les limitacions del projecte.

\section{Windows}
\subsection{Comunicació amb el dispositiu}
\subsection{Comunicació amb l'entorn gràfic}
\subsection{Servei i insta\l.lació}

\section{MacOS}
\subsection{Comunicació amb el dispositiu}
\subsection{Comunicació amb l'entorn gràfic}
\subsection{Servei i insta\l.lació}

