\chapter{Disseny del \est{software} a altres sistemes operatius}
\label{cap:software-other}

Tal com s'ha dit al principi d'aquest treball de final de grau, l'objectiu
del projecte és realitzar un programari que pugui funcionar amb la gran majoria
de sistemes operatius disponibles al mercat. Tanmateix, s'ha vist que les
llibreries per modificar aspectes gràfics dels dispositius varien molt entre
cada \est{kernel} i, per tant, la possibilitat de realitzar un únic programa que
funcioni per a tots els sistemes operatius queda descartada.

A causa de les limitacions de temps i dedicació d'aquest treball, no es podran
repetir totes les tasques executades en el capítol anterior amb la resta
de sistemes operatius. Tanmateix, es pretén aprendre el seu funcionament
per deixar un camí més senzill si mai es volgués acabar implementant un
programari per a aquests equipaments.

En aquest capítol s'explorarà la possibilitat de llegir dades d'un sensor
\acro{usb} i modificar l'orientació d'una pantalla en els dos sistemes operatius
que, juntament amb Linux, s'utilitzen en gairebé tots els ordinadors personals.
No es fixa com a objectiu crear un programa insta\l.lable per a cada un d'ells
per culpa de les limitacions del projecte.

\section{\est{Software} a Windows}

El sistema operatiu Windows acostuma a proposar un gran nombre de llibreries
a disposició del programador per poder carregar-les durant l'execució del
programa i així interactuar amb el \est{kernel}. Aquests fitxers amb
extensió \fitx{.dll} (de l'anglès \est{Dynamic Linked Library}) es poden trobar
a diversos directoris del sistema i poden ser insta\l.lats posteriorment.

Un dels grans avantatges de Windows és que els programes solen distribuir-se
juntament amb les seves dependències i llibreries, eliminant problemes de
versions mínimes suportades i compatibilitat entre diferents programes. El
gran inconvenient d'aquesta funcionalitat és que els programes ocupen més espai
de disc, i sovint s'acaba tenint algunes llibreries per duplicat.

En aquest apartat es veuran els detalls a tenir present per portar parts
de l'aplicació que interactuen directament amb el sistema operatiu a Windows.

\subsection{Comunicació amb el dispositiu}

De la mateixa manera que a Linux hi havia diversos \est{drivers} per poder
interactuar amb un perifèric des de diferents capes d'abstracció, Windows també
ofereix certa flexibilitat per treballar amb sensors. Aquesta llibreria es diu
\emph{Sensor API} i està disponible des de Windows 7 \cite{SensorMsft}.
Cal mencionar que també existeix una llibreria per interactuar amb dispositius
\acro{hid} genèrics i, a més, hi ha el projecte \est{libusb-win32}, que
vindria a ser l'analogia d'\ord|hidapi| i \ord|libusb| \cite{SensorHidMsft}.

El problema d'aquesta llibreria és que no està gaire ben documentada i té pocs
tutorials; és per això que ha sigut molt difícil aconseguir rebre dades del
dispositiu del projecte i, a diferència de Linux, no s'ha aconseguit distingir
entre dos sensors. Es deixa aquesta tasca com a una futura ampliació del projecte.

En comparació amb \ord|libiio|, aquesta llibreria ha resultat ser bastant més
complexa ja que, com que no disposa d'exemples, es creu que no s'ha realitzat
la cerca dels sensors de la forma més senzilla possible. A més a més,
aquesta llibreria està dissenyada per treballar amb llenguatges de més baix
nivell que Python, i la tasca de programació no ha sigut tan ràpida com a Linux.

\subsection{Comunicació amb l'entorn gràfic}

Canviar la rotació d'una pantalla i esbrinar la seva orientació actual no és una
tasca trivial a Windows, però tampoc és gaire complicat.
Tot i no disposar d'una llibreria específica per a les pantalles, Windows
disposa de l'estructura de C++ \acro{devmode}, definida a \fitx{wingdi.h}.
Aquest nom ve de l'acrònim anglès de \est{Device Mode} i serveix per
interactuar tant amb pantalles com amb impressores \cite{WinDevMode}.

Hi ha diverses maneres d'implementar un petit programa perquè utilitzi el
\acro{devmode} per actualitzar la rotació de la pantalla, però la majoria
dels tutorials ho acostumen a fer en el llenguatge de \ord|C#|, com és el cas
de l'exemple trobat a \cite{WinRotateOnline}. Tots els codis utilitzen
\fitx{user32.dll} per enviar l'estructura modificada al \est{kernel}.

Per tal de demostrar la viabilitat d'aquesta alternativa s'ha creat un petit
programa que inverteix l'orientació de la pantalla. Aquest codi ha funcionat
sense gaires problemes, i s'ha vist que, utilitzant la llibreria \ord|ctypes|,
es podria passar fàcilment el codi creat a Python.

\subsection{Execució en segon pla i insta\l.lació}

Finalment, un cop s'aconsegueixi que ambdues parts del programa funcionin
correctament, es voldrà distribuir l'aplicació als usuaris. Els sistemes de
Windows disposen d'un tipus de fitxer amb extensió \fitx{.msi} (de l'anglès
\est{MicroSoft Installer}) per insta\l.lar paquets (una analogia perfecta dels
paquets de Debian).

Un fitxer \fitx{.msi} té una estructura molt complexa; és per això que es recomana
utilitzar una aplicació externa que transformi un codi més llegible al fitxer
d'insta\l.lació. El programa gratuït més popular és \est{Wix}. A la seva pàgina
web hi ha un seguit de tutorials molt ben detallats sobre com crear la interfície
d'insta\l.lació i com indicar al sistema quins fitxers s'han de copiar i a on
\cite{Wix}.

\est{Wix} també ofereix la possibilitat d'insta\l.lar serveis de sistema o
d'usuari afegint unes quantes línies al fitxer de configuració de format
\acro{xml}. A Windows és molt més senzill alternar entre un servei de sistema
i un servei d'usuari i, per tant, és molt probable que tots els problemes que hi ha
hagut a Linux perquè funcioni \est{xrandr} en un servei de sistema no hi
seran en l'equivalent de Windows.

Finalment, de la mateixa forma que en els sistemes basats en Linux es pot crear
un servidor \acro{apt}, a Windows hi ha el \est{Windows Package Manager}, que
permet insta\l.lar aplicacions mitjançant l'ordre \ord|winget|. Aquest gestor
és bastant recent i poc conegut; és per això que no s'acostuma a utilitzar. La
majoria d'usuaris de Windows estan acostumats a haver de descarregar manualment
l'insta\l.lador de la pàgina web de l'aplicació i no obren gairebé mai la línia
d'ordres.

Tot i això, publicar el paquet creat al \est{Windows Package Manager} pot ser positiu
per facilitar la insta\l.lació si mai no es disposa d'un navegador o per si es vol
automatitzar el procés. A \cite{WPM} hi ha totes les instruccions per
penjar l'aplicació a la plataforma de Microsoft, i no sembla que sigui un
procés gaire llarg: penjar el fitxer \fitx{.msi}, omplir un parell de formularis
i confirmar una adreça de correu electrònic.

\section{\est{Software} a MacOS}

Finalment, s'investigarà la viabilitat de portar l'aplicació a un \est{MacBook},
la gamma de portàtils d'Apple que utilitza el sistema operatiu MacOS. Aquest
sistema operatiu compleix les directives d'\acro{unix} juntament amb els
sistemes basats en Linux. Aquestes directives defineixen les funcions i els
programes que els sistemes operatius han d'implementar perquè les aplicacions
s'hi puguin comunicar. Gràcies a \acro{unix}, una aplicació dissenyada per Linux,
és molt probable que també pugui funcionar per MacOS.

El problema és que l'aplicació que s'ha dissenyat en el marc d'aquest \acro{tfg}
no utilitza només funcions i llibreries d'\acro{unix}, sinó que també utilitza
llibreries específiques de Linux, com \acro{iio} o \est{xrandr}. Això
impossibilita la compatibilitat de l'aplicació a MacOS. En aquest capítol es
mostren alternatives a aquestes dues llibreries per aconseguir executar
l'aplicació en un \est{MacBook}.

\subsection{Comunicació amb el dispositiu}

MacOS també disposa de \ord|libusb| i \ord|hidapi|, però si es vol treballar en
una capa similar a la que s'ha estat treballant amb els altres sistemes
operatius, s'hauria de fer servir \est{\acro{Hid} Driver Kit}: és el seguit
d'eines que proporciona Apple per crear \est{drivers} de dispositius,
enfocades a dispositius \acro{hid}. No s'ha trobat cap eina específica per als
sensors.

A \cite{HidDriverKit} hi ha tota la documentació referent a \est{\acro{Hid Driver Kit}}.
Hi ha exemples que ensenyen com crear \est{drivers} per a teclats i ratolins,
que es poden extrapolar fàcilment al sensor d'aquest projecte. Per
demostrar la viabilitat d'aquesta eina, s'ha dissenyat un petit codi per
llegir les dades del sensor, que no es desvia gaire dels exemples referenciats.

El llenguatge utilitzat és C++, però també es pot fer servir \est{Objective C} o
Python per desenvolupar aquest programari.

\subsection{Comunicació amb l'entorn gràfic}

MacOS no disposa de cap \acro{api} per actualitzar l'orientació
d'una pantalla de forma directa. Tanmateix, Apple va dissenyar un llenguatge anomenat
\est{AppleScript}, que permet interactuar amb les aplicacions del sistema
mitjançant petits \est{scripts} \cite{AppleScript}.

Aquest llenguatge és únic d'Apple i no és gaire complicat de dominar, però
també és bastant susceptible a actualitzacions del sistema: sol ser complicat
crear un programa d'\est{AppleScript} que duri diverses versions sense
necessitar cap canvi. Així doncs, s'haurà de tenir present que el manteniment
d'aquest programari esdevindria més costós si s'utilitzés aquesta alternativa.

Una altra opció és utilitzar algun paquet ja existent, creat per la comunitat.
S'ha trobat el projecte \est{Fb-Rotate} \cite{FBRotate}, que utilitza una
\acro{api} de més baix nivell, anomenada \est{IOKit}, per arribar fins a
la configuració de la pantalla. Tanmateix, aquest projecte sembla bastant
abandonat, i perilla que un futur canvi en el \est{kernel} de MacOS faci que
deixi de funcionar.

Les dues alternatives són igual de viables, i cadascuna té diversos punts
forts i punts febles. S'ha aconseguit girar la pantalla en tots dos casos mitjançant
codis força senzills. Per tant, no hi ha gaires diferències respecte al cost
de desenvolupament.

\subsection{Execució en segon pla i insta\l.lació}

L'execució de programes en segon pla a MacOS es fa mitjançant l'eina
\est{launchd}. Es poden afegir serveis directament des de l'eina \est{Services}
a través d'una interfície gràfica, però també es permet afegir-ne mitjançant un
petit codi d'\est{AppleScript} \cite{AppleService}.

Tanmateix, es pot configurar un servei automàticament en el moment
d'insta\l.lació d'un paquet. L'extensió de fitxers per a paquets de MacOS és
\fitx{.pkg} i, de la mateixa forma que amb la resta de sistemes operatius,
es recomana utilitzar alguna eina externa per crear-los de manera senzilla.
En aquest cas, es recomana l'eina de \cite{ApplePackage} que, mitjançant un
seguit d'entrades a través de la línia d'ordres, genera el fitxer \fitx{.pkg}
de l'aplicació.

Finalment, també es voldrà distribuir aquesta aplicació. Apple disposa de la
seva pròpia botiga d'aplicacions, l'\est{App Store}, però s'ha de pagar una
quota anual per penjar l'aplicació. En cas que no es vulgui assumir
aquest cost, també es pot penjar a \est{HomeBrew}, una
analogia a \acro{apt} amb l'excepció que aquesta no rep el suport oficial
d'Apple (ha sigut una eina creada per la comunitat). \est{HomeBrew} permet
insta\l.lar aplicacions des de la línia d'ordres i és molt popular
entre els usuaris de MacOS.

\section{Alternativa compatible amb tots els sistemes operatius}
\label{sec:winmac-conclusions}

Tal com s'ha vist en aquest capítol, portar l'aplicació a la resta de sistemes
operatius és una tasca repetitiva i tediosa. S'ha intentat desenvolupar el codi
perquè sigui el màxim compatible amb els estàndards del sistema operatiu en
qüestió, i això ha fet que cada implementació divergís l'una de l'altra. Un clar
exemple és l'ús de \ord|libiio|, que només és disponible, de moment, en sistemes
basats en Linux.

A la taula \ref{tab:os-comparison} es mostren les diferents eines
utilitzades en funció del sistema operatiu. Com es pot veure, s'haurien de dominar
moltes més tecnologies per aconseguir que l'aplicació d'aquest projecte arribés
a la majoria d'usuaris, i els costos de desenvolupament i de manteniment
augmentarien considerablement.

\begin{figure}[ht]
    \centering
    \begin{tabular}{p{0.2\linewidth} | >{\centering\arraybackslash}m{0.2\linewidth} >{\centering\arraybackslash}m{0.2\linewidth} >{\centering\arraybackslash}m{0.2\linewidth}}
        \toprule
        \textbf{Tasca} & \textbf{Linux (Debian)}  & \textbf{Windows} & \textbf{MacOS} \\
        \midrule

        Lectura del sensor & \est{libiio} & \est{SensorApi} & \est{\acro{Hid} Driver Kit} \\
        Rotació de la pantalla & \est{xrandr} & \est{\acro{Devmode}} & \est{AppleScript} / \est{IOKit} \\
        Execució en segon pla & \est{systemd} & \est{Services} & \est{AppleScript} / \est{launchd} \\
        Extensió de fitxer del paquet & \fitx{.deb} & \fitx{.msi} & \fitx{.pkg} \\
        Eina per crear el paquet & \est{debuild} & \est{Wix} & \est{MacOS Installer Builder} \\
        Distribució del paquet & \acro{Apt} & \est{Windows Package Manager} & \est{App Store} / \est{HomeBrew} \\

        \bottomrule
    \end{tabular}
    \caption{Tecnologies utilitzades en funció del sistema operatiu.}
    \label{tab:os-comparison}
\end{figure}


En aquest apartat es vol proposar una alternativa de disseny del programari. En
aquest cas, no es buscarà treballar amb la capa d'abstracció més alta, sinó amb
la que estigui disponible en el màxim nombre de sistemes operatius. En el cas
d'aquest projecte, la capa en qüestió seria \ord|libusb|, tal com s'ha explicat
en l'apartat \ref{subsubsec:libusb}. D'aquesta manera, en crear un programa que
utilitzi \ord|libusb| i no \ord|libiio|, s'aconsegueix que també pugui funcionar
a Windows i MacOS.

Dissenyar el programari per treballar amb \ord|libusb| requereix acomplir
les tasques que realitzaven \ord|libhid| i \ord|libiio|. Això significarà que
probablement quedarà un codi resultant més llarg i complex; però en comptes d'haver-hi
tres programes, només n'hi haurà un. Haver de mantenir només una peça de programari
també ajudaria a reduir les hores de dedicació al manteniment del projecte i
tots els costos associats.

Fer que la part de la rotació de la pantalla sigui compatible amb tots els sistemes
operatius no serà una tasca igual de senzilla, ja que encara no existeix cap
llibreria o aplicació disponible per a diferents sistemes operatius. Tanmateix,
es pot crear, per exemple, un mòdul de Python amb tres implementacions
diferents: una per Windows, una per MacOS i una per Linux. En funció del
sistema operatiu es carrega una implementació o una altra i, com que totes
compartirien la mateixa \acro{api}, el programa podria funcionar correctament en
tots els entorns.

Ara que el projecte ja està fet, no es durà a terme aquest canvi, ja que suposaria
començar de nou una de les parts més grans del \acro{tfg}. Tanmateix, aquestes
conclusions es tindran en compte per a futurs treballs o projectes.
