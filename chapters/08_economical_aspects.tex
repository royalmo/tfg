\chapter{Estudi econòmic i comercialització}
\label{cap:economia}

Aquest capítol té per objectiu simular el llançament del producte al mercat.
Hi ha diversos aspectes a tenir en compte, que es trobaran detallats en
diferents apartats.

\section{Determinació del preu del producte}

El primer aspecte important és el preu del producte. Per poder definir un
preu de venda, primer s'ha de tenir present quin és el preu de cost. Durant
el transcurs del projecte s'ha produït una tirada de cinc unitats amb el
proveïdor \acro{jlcpcb}. A la taula \ref{tab:jlcpcb} hi ha el detall dels
costos de la tirada.

\begin{table}[ht]
    \centering
    \begin{tabular}{lc}
        \toprule
        \textbf{Concepte}           & \textbf{Cost}    \\
        \midrule
        Impressió de la \acro{pcb}  & \SI{1.87}{\EUR}  \\
        Muntatge de la \acro{pcb}   & \SI{10.21}{\EUR} \\
        Impressió 3D                & \SI{28}{\EUR}    \\
        Cable \acro{usb2} A-C       & \SI{5}{\EUR}     \\
        Enviament 15 dies           & \SI{1.90}{\EUR}  \\
        Components (total)          & \SI{62.47}{\EUR} \\
        - C1: \SI[round-mode=places,round-precision=1]{4.7}{\micro\farad} (5 u.)         & \SI{0.65}{\EUR}  \\
        - C2,C4: \SI[round-mode=places,round-precision=0]{10}{\micro\farad} (10 u.)      & \SI{0.18}{\EUR}  \\
        - C3,C5,C6: \SI[round-mode=places,round-precision=1]{0.1}{\micro\farad} (15 u.)  & \SI{0.08}{\EUR}  \\
        - C7: \SI[round-mode=places,round-precision=0]{2200}{\pico\farad} (5 u.)         & \SI{0.17}{\EUR}  \\
        - R1,R2: \SI[round-mode=places,round-precision=1]{5.1}{\kilo\ohm} (10 u.)        & \SI{0.19}{\EUR}  \\
        - R3,R4: \SI[round-mode=places,round-precision=0]{68}{\ohm} (5 u.)               & \SI{0.20}{\EUR}  \\
        - R5: \SI[round-mode=places,round-precision=1]{1.5}{\kilo\ohm} (5 u.)            & \SI{0.01}{\EUR}  \\
        - R6,R7: \SI[round-mode=places,round-precision=1]{4.7}{\kilo\ohm} (10 u.)        & \SI{0.19}{\EUR}  \\
        - D1,D2: \SI[round-mode=places,round-precision=1]{3.3}{\volt} (14 u.)            & \SI{0.52}{\EUR}  \\
        - J1: \acro{USBC} (7 u.)            & \SI{0.60}{\EUR}  \\
        - U1: \acro{MCP1703T} (5 u.)        & \SI{3.43}{\EUR}  \\
        - U2: \acro{AtTiny85} (5 u.)        & \SI{18.89}{\EUR} \\
        - U3: \acro{MPU6050} (5 u.)         & \SI{37.36}{\EUR} \\
        \midrule
        \textbf{TOTAL (IVA Inclòs)} & \textbf{\SI{109.45}{\EUR}} \\
        \bottomrule
    \end{tabular}
    \caption{Cost de produir una tirada de cinc dispositius.}
    \label{tab:jlcpcb}
\end{table}


A la taula també hi ha els costos d'altres parts, com ara la impressió
3D i el connector \acro{usb}. En els dos casos s'ha aproximat el preu, perquè,
o bé ja es disposava d'aquest material (i no se n'ha comprat), o bé l'escola
l'ha proporcionat gratuïtament.

Tal com es pot veure, hi ha certs components en els quals la quantitat més petita
d'unitats que es poden comprar és superior al que es necessitava. Tenint aquest
factor en ment i sabent que els preus solen disminuir quan s'augmenten les
quantitats, s'ha decidit simular la compra de 10000 unitats.

Per fer aquesta simulació s'ha cercat als diferents portals web (de les
impressions 3D, dels components, \dots) el preu aproximat per a aquestes
quantitats. Com és evident, els nombres obtinguts són molt aproximats i
arrodonits i, en cas de voler-ho dur a terme, s'haurien de revisar amb més
cura. A la taula
\ref{tab:pricing10k} es pot consultar el desglossat de preus per a
grans quantitats.

\begin{table}[ht]
    \centering
    \begin{tabular}{lc}
        \toprule
        \textbf{Concepte}           & \textbf{Cost} \\
        \midrule
        Impressió de la \acro{pcb}  & \SI[round-mode=places,round-precision=0]{400}{\EUR} \\
        Muntatge de la \acro{pcb}   & \SI[round-mode=places,round-precision=0]{14000}{\EUR} \\
        Impressió 3D                & \SI[round-mode=places,round-precision=0]{8100}{\EUR} \\
        Cable \acro{usb2} A-C       & \SI[round-mode=places,round-precision=0]{7800}{\EUR} \\
        Enviament 15 dies           & \SI{18.5}{\EUR} \\
        Components (total)          & \SI[round-mode=places,round-precision=0]{69200}{\EUR} \\
        \midrule
        \textbf{TOTAL (IVA Inclòs)} & \textbf{$\approx\SI[round-mode=places,round-precision=0]{100000}{\EUR}$}  \\
        \midrule  
        \textbf{Preu per unitat}    & \textbf{$\approx\SI[round-mode=places,round-precision=0]{10}{\EUR}$} \\
        \bottomrule
    \end{tabular}
    \caption{Cost de produir una tirada de
        \num[round-mode=places,round-precision=0]{10000} dispositius.}
    \label{tab:pricing10k}
\end{table}


Així doncs, si es produís aquest producte en massa, el preu de cost rondaria
els \SI[round-mode=places,round-precision=0]{10}{\EUR}.
Tanmateix, el preu de venda sempre hauria de ser més elevat que
el preu de cost. El marge que es defineixi pot dependre de diversos factors.
A continuació es detallen alguns aspectes que s'haurien de tenir presents
a l'hora de fixar un preu:

\begin{itemize}
    \item Primer de tot, s'hauria de realitzar un estudi de mercat per veure
    si hi ha productes similars i quin preu i èxit tenen. En general, el nou
    producte hauria de tenir un preu similar als ja existents.
    \item Es pot complementar l'estudi anterior amb un estudi del \est{target},
    és a dir, preguntar als possibles compradors quin preu estan disposats a
    pagar pel producte.
    \item S'ha de valorar si es vol tenir en compte els costos de
    desenvolupament. Si aquest producte no s'hagués creat en el marc d'un \acro{tfg},
    s'hauria hagut de pagar a la persona que ho desenvolupés en funció del
    temps dedicat. Per saber quin percentatge s'ha de dedicar al salari d'aquesta
    persona, s'ha de fer una estimació del nombre d'unitats que es vendran.
    \item S'ha d'afegir al preu de cost les despeses que es puguin ocasionar
    durant la publicitat i venda del producte.
    \item S'ha de determinar, si escau, un marge de benefici pels inversors
    del projecte.
    \item Finalment, es pot arrodonir el preu final en funció d'altres aspectes,
    d'entre els quals hi ha el psicològic: un producte a
    \SI[round-mode=places,round-precision=0]{19}{\EUR} és més cridaner que un
    altre a \SI[round-mode=places,round-precision=0]{20}{\EUR}
    \cite{kumar2017impact}.
\end{itemize}

S'ha realitzat un còmput d'hores de dedicació al desenvolupament del projecte i
s'ha aproximat a 150 hores el temps de recerca i creació del producte.
S'haurien de sumar les hores dedicades a completar el capítol
\ref{cap:software-other} al total anterior, però per fer un estudi orientatiu
es considera suficient fer servir aquest nombre.

Suposant un preu brut per hora de treball de
\SI[round-mode=places,round-precision=0]{30}{\EUR\per\hour}, el cost de
desenvolupament seria de \SI[round-mode=places,round-precision=0]{4500}{\EUR}.
En la tirada de dispositius que es preveu fer a la taula \ref{tab:pricing10k},
aquest cost suposaria menys de 50 cèntims per dispositiu venut. Tot i que es
decidís posar un preu per hora més elevat o es contemplés el temps dedicat a
redactar aquest document, el cost de desenvolupament continuaria sent molt menor
al dels materials necessaris per crear cada unitat.

Així doncs, arrodonint a l'alça, es pot fer el supòsit que el preu més baix al
que es pot vendre el producte és de
\SI[round-mode=places,round-precision=0]{12}{\EUR}. Només falta afegir un marge
de benefici per als inversors o propietaris de la patent.
Tenint presents les recomanacions anteriors, es podria valorar que
un bon preu de venda podria ser
\SI{19.8}{\EUR}. Aquest preu agafa per referència el
producte acabat completament, és a dir, totes les imperfeccions i millores que
se citaran al capítol \ref{cap:future_work} haurien d'estar acabades.

Aquest marge afegit al preu final ofereix la possibilitat de realitzar
descomptes i poder vendre el producte a un preu inferior en ocasions especials
sense perdre-hi diners. De la mateixa manera, contempla petites variacions en
els costos de producció i hores de desenvolupament addicionals per corregir
possibles errors. També es pot afegir en aquest marge un petit servei per
gestionar les garanties o oferir un suport tècnic.

\section{Comercialització del producte}

Un cop se sap el preu del producte i s'està disposat a vendre'l, es pot
començar a comercialitzar-lo. Tanmateix, encara queden alguns aspectes pendents
per definir, que es detallen en aquesta secció.

\subsection{Imatge comercial del projecte}

Donar una imatge al producte, al projecte i a l'empresa és una part essencial
durant el procés de comercialització. Per aquest motiu s'ha de destinar una
partida del pressupost en tasques d'imatge.

La més important sol ser tenir una pàgina web on hi hagi tota la informació
necessària de cara al client. És important que hi hagi enllaços per comprar el
producte, veure valoracions d'altra gent i contactar amb l'empresa.

Generar un logo o icona per a les aplicacions i altra documentació també és una
bona forma d'associar una imatge al producte. També és important que els clients
recordin el producte per la facilitat d'ús (mitjançant tutorials) i no pels
maldecaps que els hi pugui haver generat.

Tot i que no s'ha previst vendre en un futur pròxim el producte, s'ha creat una
petita icona per a l'aplicació que es pot veure a la Figura \ref{fig:icon},
i s'ha creat una pàgina web estàtica molt
senzilla. Ambdues coses es poden consultar visitant
\url{https://gyroscreen.ericroy.net} o el repositori del projecte.

\begin{figure}[ht]
    \centering
    \includegraphics[width=0.2\textwidth]{images/gyroscreen.png}
    \caption{Icona del projecte.}
    \label{fig:icon}
\end{figure}

A la placa de circuit imprès s'ha afegit l'enllaç de la pàgina web anterior.
S'ha de tenir en compte que, si el projecte creixés més, seria convenient tenir
un domini propi per al producte. Tanmateix, el fet de pertànyer a un domini
pare fa que hereti la seva imatge. Un clar exemple és la pàgina web de
la titulació que s'està cursant, \url{https://ocwitic.epsem.upc.edu}, que
hereta el nom de la \acro{upc}.

\subsection{Llicències del codi}

Un punt que s'ha d'acabar de debatre abans de llançar el producte és si es
publicarà el codi font. Tal com s'ha vist en les condicions de la llibreria
\acro{v-usb}, el \est{firmware} s'hauria de penjar amb la llicència \acro{gpl-3},
excepte si es volgués pagar una llicència comercial que, en funció dels dispositius
que es vulguin produir, s'hauria de pagar més o menys \cite{VusbLicensing}.

Per simplificar les coses i evitar possibles problemes legals, s'ha decidit
publicar totes les parts del sistema sota la mateixa llicència que \acro{v-usb}:
\acro{gpl-3+}. Tanmateix, s'ha deixat clar en la documentació del projecte que
es pot contactar amb l'autor del projecte per si es vol pactar una llicència menys
restrictiva.

\subsection{Llicència USB}

Tal com s'ha comentat en l'apartat \ref{subsec:usb-if}, durant el desenvolupament
del projecte s'han utilitzat codis \acro{vid} i \acro{pid} arbitraris, tenint
cura que no fossin de productes coneguts. Ara bé, en treure el producte al
mercat és convenient reservar una parella de codis per a aquest dispositiu.

En el mateix apartat s'explica que hi ha entitats que, sense cobrar res, assignen
identificadors a projectes de codi i maquinari obert. Un d'aquests és
\texttt{pid.codes}, que utilitza el sistema co\l.laboratiu de \est{GitHub} per
acceptar noves peticions \cite{PidCodes}.

Així doncs, s'ha so\l.licitat el \acro{vid/pid} \texttt{0x1209}-\texttt{0xF4F4}
per al projecte \est{Gyroscreen}. Fins poc abans del moment de l'entrega del
treball encara no hi havia hagut resposta per part de la persona que manté el
projecte. Cal tenir en compte que qui gestiona \texttt{pid.codes} no guanya res
a canvi, només hi perd temps i diners \cite{PidCodesFaq} i, per això, no està
sempre pendent de les noves peticions.

Tanmateix, el 21 de maig de 2024, Scott Shawcroft va aprovar la parella de codis
\acro{pid/vid} per ser reservades per a aquest projecte \cite{PidCodesPR}.
Així doncs, queda oficial que la parella \texttt{0x1209}-\texttt{0xF4F4} és
d'ús exclusiu per a dispositius \est{Gyroscreen}, és a dir, per a aquest
projecte.

\subsection{Inversió inicial de capital}

Finalment, s'haurà de fer una inversió de diners per poder produir els
dispositius. En funció del nombre de dispositius, la inversió serà més o menys
important, igual que els beneficis.

Trobar inversors no és tasca fàcil. Per això, si es necessita suport
econòmic, és molt important que s'acabi dedicant un temps prudencial a preparar
possibles visites o entrevistes amb persones que estiguin disposades a confiar
amb projecte.

També es poden utilitzar plataformes web co\l.laboratives, com per exemple
\est{kickstarter}, que agrupa petites aportacions de molts usuaris per
aconseguir suficient capital per a un projecte gran \cite{Kickstarter}.
