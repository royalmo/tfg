\chapter{Objectius}
\label{cap:objectius}

\section{Objectius generals}

L'objectiu principal d'aquest treball de fi de grau és implementar un dispositiu
de dimensions reduïdes, que s'enganxarà darrere el monitor i es connectarà
a l'ordinador mitjançant un cable \acro{usb}. Aquest dispositiu contindrà els
sensors i altres components necessaris per poder informar l'ordinador sobre
l'orientació actual de la pantalla. A través d'un programari propi i de la
interfície de programació que proporciona l'entorn gràfic, es canviarà
l'orientació de la pantalla en funció de les dades rebudes del dispositiu. Es
buscarà sempre la màxima compatibilitat possible, facilitat d'insta\l.lació i
ús quotidià, i un preu de construcció raonable.


\section{Objectius específics}

Aquest treball de fi de grau es podria estendre molt més enllà dels pocs mesos
de temps dels que es disposa. Per això, es definiran uns objectius essencials
per considerar el projecte completat, i uns objectius addicionals o
ampliacions que, tot i no ser troncals pel projecte, poden completar o
millorar el sistema.

Objectius essencials:
\begin{itemize}
    \item Fer una recerca sobre la tecnologia i pràctiques actuals sobre el 
    procés de disseny de dispositius \acro{usb}.
    \item Escollir els components electrònics del dispositiu i dissenyar una
    placa de circuit imprès. S'imprimiran unes quantes plaques per poder fer
    proves del sistema complet.
    \item Definir la forma en què l'usuari podrà insta\l.lar i configurar
    el sistema.
    \item Adaptar una implementació o implementar el programari que comunicarà
    el dispositiu i l'ordinador. Aquesta aplicació haurà de complir els
    requisits d'interacció de l'usuari definits en el punt anterior. En aquest
    punt es centrarà només en l'entorn de GNU/Linux.
\end{itemize}

Objectius addicionals:
\begin{itemize}
    \item Adaptar el sistema a Windows i MacOS fent-lo, doncs, compatible amb la
    gran majoria de sistemes operatius del mercat.
    \item Un cop es tingui la placa, dissenyar i construir una carcassa pel
    dispositiu.
    \item Dissenyar i implementar una interfície d'usuari senzilla i amable
    que permeti configurar el dispositiu.
    \item Realitzar un estudi econòmic per a una possible comercialització del
    projecte.
    \item Dissenyar una pàgina web senzilla per promocionar el projecte.
    \item Preparar el sistema per poder acceptar més d'un dispositiu
    simultàniament.
\end{itemize}

\section{Estructura general del projecte}

S'ha considerat que, per a comprendre millor el projecte i les decisions preses,
és convenient tenir una idea general del projecte. D'aquesta forma, es pot veure
quines parts són les mès crítiques o complicades. Evidentment, al no haver fet
cap recerca encara l'arquitectura tindrà moltes ambigüitats.

A la figura \ref{fig:project-structure} es pot contemplar l'arquitectura inicial
del projecte. S'ha determinat les peces que formen aquest puzzle a partir d'una
recerca molt superficial, i es preten afirmar la certesa durant el transcurs
d'aquest document.

\begin{figure}[ht]
    \centering

    \begin{tikzpicture}

    
    \end{tikzpicture}

    \caption{Esquema simplificat de l'estructura del projecte.}
    \label{fig:project-structure}
\end{figure}


Tal i com s'ha comentat en l'apartat anterior, es desenvoluparà el maquinari per
a funcionar correctament en sistemes Linux, i la portabilitat a la resta de
sistemes operatius es considera una possible ampliació. Tot i això, l'esquema
anterior serveix per a tots els sistemes operatius, donat que tots disposen de
\est{Drivers}, entorn gràfic o gestor de serveis. Només varia la implementació i
la forma amb la que l'aplicació hi haurà de interactuar.

No es defineix quin protocol de comunicació hi haurà entre el microcontrolador
i el sensor, donat que això pot canviar en funció del sensor que s'acabi
escollint. De la mateixa manera, tampoc es defineix com es comunicaran les
diverses parts de programari que hi haurà a l'ordinador (en blau a l'esquema).

El que sí que es pot definir és la forma d'interacció de l'usuari amb el
programa. Tret de l'etapa d'insta\l.lació, l'usuari modificarà un fitxer per a
poder configurar el programa. Quan es vulguin aplicar els canvis, aturar,
executar o reiniciar el programà es farà mitjançant un gestor de serveis. En el
cas dels sistemes basats en Linux, aquest serà \est{systemd}.

Tenint doncs l'estructura del sistema en ment resulta més senzill veure per on
es pot començar el treball.
