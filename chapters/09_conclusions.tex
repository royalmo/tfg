\chapter{Conclusions}

En aquest treball de final de grau s'ha creat un sistema que, mitjançant la
connexió \acro{usb} i un sensor d'acceleració tridimencional, pot detectar quan
es gira un monitor de l'ordinador i ordenar al sistema operatiu que canvii
l'orientació d'aquella pantalla. Tal i com s'ha comentat al principi del
document, la importància del treball rau en el procés de creació del dispositiu,
més que en el resultat final. S'ha optat per a crear un dispositiu des de zero,
utilitzant llibreries i protocols ja establerts, però dissenyant totes les peces
del sistema per un mateix.

En primer lloc, s'ha cercat informació sobre el protocol \acro{usb} i la
implementació de petits dispositius. Fruit d'aquesta recerca, s'ha continuat
investigant en les taules d'ús \acro{hid} i la llibreria \acro{v-usb}.
Para\l.lelament, s'ha seleccionat el maquinari del sistema, entre els quals hi
havia el sensor \est{MPU6050}. Aquest sensor es comunica amb el protocol
\acro{i2c}, que també s'ha hagut d'aprendre i implementar.

Després de la recerca es va dissenyar el circuit electrònic del dispositiu,
la placa de circuit imprès, i una carcassa creada amb una impressora 3D.
Tenint tot el disseny físic fet, es va decidir produir-ne 5 unitats.

Un cop la part del dispositiu enllestida s'ha investigat com crear el
programari necessari per a utilitzar el dispositiu en un entorn Linux. S'ha
investigat diverses llibreries i finalment s'ha creat un codi en Python que
comunica totes les peces del sistema. Un cop el codi era funcional, es va
procedir a convertir-lo en una aplicació robusta i insta\l.lable, creant una
pàgina de manual, un fitxer de configuració, un fitxer de servei de sistema,
dependències, entre moltes altres coses.

El següent pas va ser repetir la tasca anterior per a la resta de sistemes
operatius. En aquest cas, l'objectiu d'aconseguir una aplicació robusta no s'ha
acabat assolint degut a les limitacions de temps del projecte, però sí que s'ha
pogut demostrar que també és possible implementar-ho, i que només suposaria
unes poques setmanes de treball més per a aconseguir resultats igual de bons
que amb l'entorn Linux.

En aquest projecte també s'ha posat èmfasi en la part legal de la programació.
Per una banda, s'ha tingut molta cura a l'hora de llicenciar tots els continguts
creats, i que aquetes llicències no incompleixin les condicions de les
llicències de les llibreries i altres projectes utilitzats. El clar exemple és
el compliment de la llicència de la llibreria \acro{v-usb}. Per altra banda,
s'ha certificat el projecte davant entitats reconegudes, com l'\acro{oshwa}, i
s'ha fet tot el possible (dins del pressupost d'un projecte d'aquestes
dimensions) per a obtenir un identificador \acro{vid} i \acro{pid} únics, per a
evitar co\l.lisions amb altres dispositius \acro{usb}.

Una de les conclusions més inesperades per part de l'autor és que s'ha
observat que la documentació existent sobre la creació de dispositius
\acro{usb} no és tan abundant com amb la resta de camps de la informàtica.
Qualsevol diria que, pel fet de tenir connectors \acro{usb} a tot arreu,
els tutorials i documentació no haurien d'escassejar. De
fet, s'ha observat fins i tot errades en les llibreries d'aquest nivell, i s'ha
publicat correccions a aquestes. A data de la publicació d'aquest treball encara
no s'ha acceptat la contribució que s'ha fet a la llibreria \est{libiio}.

Si aquest projecte ha arribat a bon port és, en gran part, perquè s'ha sabut
detectar les errades a temps per a corregir-les. Tanmateix, el fet de iterar
diverses vegades en el mateix sistema i proposar millores que
impliquen refer o modificar considerablement el treball que ja s'havia
considerat definitiu ha fet molt complicada la tasca de redactar aquesta
documentació seguint un fil conductor lineal. En projectes de diverses persones
o més grans que aquest no hagués estat possible incorporar canvis trencadors
a mig projecte, s'hagués acabat el prototip i, un cop anotades les conclusions
i millores, s'hagués dissenyat un de nou amb aquests canvis. La metodologia
àgil utilitzada per a aquest projecte té el benefici que no s'ha dedicat més
temps del necessari a prototips que no són definitius ni aporten res al
producte final. Tanmateix, ha aportat la complexitat de no poder dividir tan
fàcilment el projecte per parts, ja que han estat totes molt entrellaçades.

Pel que fa a l'àmbit més personal, aquest treball ha estat una experiència molt
enriquidora. No solament s'ha aconseguit treballar en totes les branques de la
titulació, sinó que s'ha posat èmfasi en les parts en les que hi havia més
interès, però també més desconeixement. La falta de coneixements ha conduït a
petits errors, però també és una bona forma d'adquirir experiència i evitar
cometre errades en un futur.
