\chapter{Firmware}

En aquest capítol es detallarà tot el referent al codi que anirà dintre de la
placa de circuit imprès realitzada en el capítol anterior. Degut que les
connexions entre la placa creada i l'entorn de proves amb la placa \est{Digispark}
i el mòdul \est{GY291} són idèntiques, el \est{firmware} que es crei servirà
per les dues plaques.

Tal i com s'ha dit en l'apartat \ref{subsec:hw_digispark}, es va descobrir
el projecte \est{i2c\_on\_littlewire}, que permet comunicar-se directament amb
dispositius \acro{i2c} des de l'ordinador. En aquest apartat també es destaquen
els motius pel que s'ha descartat aquesta possiblitat, però tot i això cal
mencionar que es va crear un petit codi de C per a demostrar que la comunicació
era possible. Aquest codi es va basar en els exemples disponibles a (CITE).

\section{Entorn de treball de desenvolupament}

Per a poder desenvolupar amb comoditat el \est{firmware} del projecte és crucial
facilitar la tasca de provar les modificacions fetes. Per aquest motiu, s'ha
planejat tot un entorn de treball que es detalla en aquest apartat.

\subsection{\est{Bootloader}}
\label{subsec:bootloader}

Hi ha diverses formes de programar el microcontrolador, però la més adequada
per al sistema en qüestió, i tenint present que es voldrà programar el dispositiu
moltes vegades, és utilitzar un \est{bootloader} que identifiqui el dispositiu
com una placa programable durant els primers segons d'operació. Si al cap de
poc temps que el dispositiu no estigui alimentat l'usuari no ha intentat
programar-hi res a través de l'ordinador, s'executarà el programa principal
del microcontrolador.

Aquest \est{bootloader} que s'ha descrit és exactament el que hi ha en la
placa \est{Digispark} per defecte. Com és evident, en un producte definitiu no
interessa posar un \est{bootloader} que retardi el començament del programa
uns segons més, però per a fer proves valdrà la pena aquesta espera.

Programar el \est{Digispark} o una placa equivalent com la d'aquest projecte
utilitzant el \est{bootloader} definit anteriorment és una tasca molt senzilla.
Un cop es disposi del fitxer \verb|.hex| que es vol programar al
microcontrolador, es pot utilitzar programes com \est{avrdude} o
\est{micronucleus} per a que, mentre s'estigui executant el bootloader, es 
reprogrami el xip. (CITE)

Per a progrmar la placa amb un sistema basat en Linux s'ha de realitzar una
tasca extra: els permisos. Un usuari administrador podrà programar la placa
sense cap problema, però executar com a \est{root} un programa sempre implica
assumir riscos (CITE). Per això el més recomanat és crear una regla \acro{udev}
per al dispositiu en qüestió, fent que sigui accessible per a tots els usuaris.
Es pot consultar un tutorial detallat a (CITE).

\subsection{\est{Makefile}}

Un cop es sap com programar la placa, el següent pas a seguir és crear un
\est{Makefile} per a poder compilar i programar amb una única comanda.
No és la primera ni la segona vegada que s'utilitza l'eina \est{make} en
aquesta titulació però sí que és la primera vegada que s'utilitza amb una placa
que no sigui un Arduino. Tanmateix, no hi ha gaire diferència, tret de la
part de programar mencionada en l'apartat anterior.

S'ha preparat i documentat el \est{Makefile} com es mereix: s'ha posat un menú
d'ajuda, s'ha posat l'opció per a programar els fusibles del microcontrolador
(aquesta part només s'ha de fer una sola vegada), s'ha posat opcions per a
eliminar tots els fitxers generats i també per a poder debugar el programa.

\section{\acro{V-usb}}

Amb l'entorn de programació preparat, ja es pot començar a crear el
\est{firmware} del projecte. Es començarà per la part més complicada: la
llibreria \acro{v-usb}. Tanmateixa, aquesta té molt bona documentació i
tutorials disponibles per internet.

\subsection{Ca\l.libració de l'osci\l.loscopi}

\subsection{Identificadors \acro{usb}}
\subsection{\est{Watchdog}}

\subsection{canvi a accel3d}

\section{\acro{I2c}}

Un cop solucionada la comunicació entre el microcontrolador i l'ordinador, toca
centrar-se en la comunicació entre el microcontrolador i el sensor. Aquesta, tal
i com s'ha dit en el capítol anterior, es farà mitjançant el protocol \acro{i2c}.
Aquest protocol utilitza dues línies bidireccionals, una per les dades i una per
el rellotge, que es connecten a 0V o es deixen a l'aire (i els \est{pull-ups} la
porten a la tensió d'alimentació).

Si bé hi ha moltes llibreries per a la placa Arduino i altres microcontroladors
de l'arquitectura \acro{avr}, totes utilitzen perifèrics específics que, en el
cas de l'\acro{AtTiny85} d'aquest projecte, o estan ja en ús per \acro{v-usb}, o
no existeixen per a aquest xip. Així doncs, s'ha de aprendre el funcionament a
baix nivell del protocol i implementar-lo des de zero.

Per sort, aquest protocol no és molt extens i amb unes 300 línies de codi en C
s'ha pogut implementar el parell de funcions que es necessitava per a dur a
terme el projecte.

El protocol \acro{i2c}, com s'ha dit anteriorment, és molt versàtil i permet 
to\l.lerar dispositius amb diferents tensions d'alimentació, més dispositius
(esclaus, però també mestres) i molta escalabilitat. El mestre és qui inicia
la transmissió (per tant, dos esclaus no poden parlar entre si), però la
direcció del flux de dades pot canviar. CITE

El funcionament de \acro{i2c} es pot resumir en poques paraules: el mestre envia
pel canal una adreça de 7 bits per a saber amb quin dispositiu vol comunicar-se.
Aquests 7 bits són l'adreça del dispositiu, disponible als \est{datasheets}. Per
al sensor d'aquest projecte, l'adreça és 0x68 CITE. Després d'aquests 7 bits,
s'envia un 8è bit que dependrà de si es vol enviar dades o rebre dades del
dispositiu.

Un cop rebut el \est{acknowledgement} de l'esclau, comença la transmissió de
dades, que acabarà en funció dels valors de \est{acknowledgement} del mestre, o
si el mestre envia un senyal de stop. Es recomana llegir el \est{dataheet} del
sensor a CITE per a entendre el funcionament de \acro{i2c}, ja que està molt ben
explicat.

\subsection{Funcionament del sensor \est{MPU6050}}

Pel cas concret del sensor d'aquest projecte, hi ha un parell de normes extres
a part de les que estableix el protocol \acro{i2c} que serveixen per a comunicar-se
d'una forma fiable amb el sensor.

El sensor \est{MPU6050} defineix en el seu \est{datasheet} un seguit de
registres, que funcionen de la mateixa forma que els perifèrics en \acro{avr}:
no son registres comuns per a guardar dades, sinó que és com si hi hagués una
extensió en els perifèrics del propi microcontrolador. Així doncs, si es llegeix
el registre 0x50:0x51 REVISAR es podrà consultar el valor mesurat de temperatura
(sense processar) CITE. També es pot escriure en alguns registres, per exemple
per a activar o ca\l.librar el sensor.

Per a escriure en un registre només cal enviar per \acro{i2c} (és a dir, escriure)
l'adreça del registre i el valor a escriure. Si s'envien més bytes, el sensor
entendrà que s'han d'escriure en els registres posteriors. Pel que fa a la lectura,
s'ha d'enviar l'adreça del registre i tornar a escriure la capçalera \acro{i2c},
però en aquest cas, en mode de lectura. El sensor enviarà els continguts del
registre en qüestió. Si es decideix de seguir llegint (mitjançant un \acro{nack}),
el sensor enviarà el valor del registre posterior. CITE

Finalment, cal destacar que, pel que fa al sensor, poca configuració se li ha de
fer per a rebre dades d'acceleració. Només cal desactivar el mode \est{sleep},
que ve sempre activat per defecte per a estalviar energia, i començar a llegir
dades. Per els propòsits d'aquest projecte no és rellevant la freqüència de
dades, qualsevol serà suficient. La precisió de les dades tampoc és tan rellevant
per als calculs que s'haurà de realitzar.

\section{Programació dels dispositius de producció}

Després de resoldre alguns detalls que no tenen suficient importància per a
incloure en aquest document, el \est{firmware} ja estava llest per a 
implantar-se a molts dispositius. Tanmateix, tal i com s'ha dit a
l'apartat \ref{subsec:bootloader}, no es vol que els productes finals tinguin
un \est{bootloader}. Per tant, s'ha de cercar una nova forma de programar
els microcontroladors.

La placa \acro{AtTiny85} ve, per defecte, amb la memòria \est{flash} completament
buida, és a dir, no té ni \est{bootloader}. La forma més senzilla de poder
programar-hi alguna cosa és mitjançant \acro{isp}. El protocol
\est{In-System Programming}, també conegut sota les sigles \acro{icsp} o
\est{In-Circuit Serial Programming}, és una forma prou estandaritzada per a
programar o comunicar-se amb microcontroladors programables, sensors, o altres
dispositius del món dels sistemes encastats. (CITE)

Una comunicació per \acro{isp} necessita 6 cables, o 4 si no es tenen en compte
les línies d'alimentació. Ara bé, només queden 2 potes disponibles, i \acro{isp}
necessita d'unes potes específiques (no es poden configurar per codi, ja que
encara no s'ha programat la placa). Així doncs, la solució més sensata és
dessoldar el microxip, programar-lo en una placa a part, i tornar-lo a soldar
a la placa.

Abans de fer aquesta tasca tediosa, però, s'ha decidit consultar amb el
personal del laboratori de l'escola, que ha recomanat utilitzar una pinça com
la de la figura \ref{sdfasdf} per a accedir a les potes de l'integrat sense haver
de dessoldar-lo. Programar així el microcontrolador no és sempre una bona idea,
sobretot si hi ha elements actius que podrien causar un curtcircuit. Però al estar
la connexió \acro{usb} a l'aire (no hi ha cable connectat) i la connexió
\acro{i2c} funciona mitjançant \est{pull-ups}, no hi ha hagut cap problema per
a programar així l'integrat.

Per a programar per \acro{isp} s'ha utilitzat una placa Arduino amb el codi
\est{ArduinoISP}, disponible a la pròpia aplicació \est{ArduinoIDE}. (CITE) Un cop
l'Arduino tenia aquest codi, una simple comanda d'\est{avrdude} s'ha afegit
al \est{Makefile}, permetent la programació senzilla del dispositiu.
