\chapter{Hardware}

En aquest capítol es detallarà tot el procés de realització del maquinari per
al projecte. La base teòrica del capítol anterior pot resultar de gran utilitat
per a entendre certes decisions preses durant algunes de les tasques d'aquest
capítol.

En aquest projecte es vol obtenir una placa de circuit imprès que permeti
transferir dades d'un sensor d'acceleració a l'ordinador, mitjançant la connexió
\acro{usb}. S'enviarà a producció la placa i es crearà un encapsulat senzill per
a evitar que aquesta es pugui omplir de pols o d'altres elements de l'entorn.
Finalment, es llicenciarà el disseny davant d'una entitat certificadora.

\section{Selecció del sensor}

El primer pas per a crear la placa és escollir quin sensor utilitzar. Per a
mesurar la inclinació d'un dispositiu es sol utilitzar un acceleròmetre.
Mitjançant un càlcul que es veurà més endavant i donades unes condicions es pot
determinar l'angle en relació a la direcció de la gravetat de la Terra. CITE.
Aquestes mesures es poden complementar amb un giroscopi, que ofereix més
precisió al sistema. CITE.

Després de fer una mica de recerca sobre els productes disponibles del mercat,
i tenint present la precisió que es necessita i el pressupost que es disposa,
s'ha trobat dues alternatives viables.

\begin{listing}
    \item El sensors \est{ADXL3xx} es poden trobar a preus molt llaminers i
    comuniquen la lectura de l'acceleració en 3 dimensions mitjançant 3 línies
    analògiques. Per tant, la comunicació amb el microcontrolador serà el més
    senzilla possible. (CITE)
    \item El sensor \est{MPU6050}, en canvi, es comunica amb el microcontrolador
    mitjançant el protocol \acro{i2c}. Aquest protocol només necessita dues
    línies de comunicació, però implica implemenar una mica de lògica per a
    recuperar les dades. Tanmateix, aquest sensor també ofereix lectures de
    velocitat angular i temperatura, que podrien ser útils per a futures
    millores del projecte. (CITE)
\end{listing}

Així doncs, degut que el segon sensor és només pocs cèntims més car i ofereix
la possiblitat d'ampliar el projecte en un futur, s'utilitzarà el sensor
\est{MPU6050} per al sistema.

Aquest sensor es sol vendre amb una placa que estalvia connectar
certa electrònica i permet provar el sensor sense fer cap soldadura.
Aquest mòdul, anomenat \est{GY291}, es troba disponible a moltes botigues
d'electrònica, i s'ha decidit adquirir-ne un parell per al desenvolupament del
projecte.

TODO POSAR CIRCUIT GY291

El circuit del mòdul \est{GY291} és lliure i es pot consultar fàcilment (CITE).
Serà molt útil disposar d'aquests esquemàtics per a poder facilitar la tasca de
dissenyar el circuit.

\section{Primer disseny}

En un primer moment es va decidir per utilitzar un microcontrolador que disposés
del perifèric \acro{usb} directament al maquinari, per a evitar programar-lo tot.
Tanmateix, es veurà que no s'acabarà utilitzant aquesta versió degut a motius
que s'explicaran més endavant. Dit això, s'ha decidit conservar aquest apartat
per a comprendre millor el procés de desenvolupament del projecte.

El microcontrolador que s'utilitzarà en aquesta versió és l'\acro{AtMega32u4}.
Aquest proporciona una interfície \acro{usb} integrada, i permet utilitzar la
interfície \acro{i2c} amb suficient facilitat. S'ha decidit utilitzar un
osci\l.lador extern per a generar un rellotge de freqüència $16MHz$
al microcontrolador.

\subsection{Esquemàtic}

El disseny de l'esquemàtic es basa en el disseny del mòdul \est{GY291} i es pot
consultar a la figura \ref{todo}. Tret del que s'ha comentat en els paràgrafs
anteriors, la resta de components i connexions no deixen de ser les evidents per
a un circuit amb aquestes característiques. Tot i això, s'ha decidit posar
èmfasi en alguns detalls on s'hi ha parat més atenció:

\begin{itemize}
    \item Al costat de l'alimentació de cada integrat s'hi ha posat un
    condensador ceràmic de $1nF$ VERIFICAR!!, per a assegurar-se que la tensió
    d'entrada és prou estable.
    \item S'ha hagut de posar un regulador lineal en el ciruit, ja que el
    sensor necessita estar alimentat a $3.3V$, però l'alimentació que proporciona
    el connector \acro{usb} és de $5V$.
    \item Al utilitzar un connector de tipus C, tot i communicar-se amb
    \acro{usb2}, hi ha un parell de cables extres que determinen a quina tensió
    s'ha d'alimentar el dispositiu. Tal i com es comenta en CITE REFERENCIA,
    si es desitja utilitzar una alimentació tradicional de $5V$ és suficient
    utilitzar dos \est{pull-ups} de XX ohms TODO VERIFY.
    \item El protocol \acro{i2c} necessita que les dues línies de comunicació
    estiguin amb \est{pull-ups}. Es veurà més endavant en el document el
    funcionament d'aquest protocol.
\end{itemize}

\subsection{Placa de circuit imprès}

Un cop acabat el disseny, utilitzant també el programa KiCad, es va començar a
crear la placa de circuit imprès. El procediment és prou senzill i es sol fer
de forma iterativa: crear un disseny molt dolent i anar-hi fent millores, fins
al punt on es considera que ja està prou bé (un disseny mai estarà perfecte).

A la figura \ref{todo} es pot consultar la placa resultant. Com es pot veure,
no està del tot ben distribuida: això és perquè es va decidir canviar completament
de disseny abans d'haver-la acabat, i es va pensar que no valia la pena seguir
invertint temps amb un disseny que no veuria mai la llum.

\subsection{Problema del proper disseny}

Tal i com s'ha dit a l'apartat anterior, no es va acabar de dissenyar la placa
de circuit imprès degut al redisseny del \est{hardware}. El motiu es ben senzill,
no s'està utilitzant el microcontrolador adequat per al projecte.

Tal i com s'ha vist a l'esquemàtic, el microcontrolador només necessita 4 potes
per a informació (2 per \acro{usb} i 2 per \acro{i2c}). Això significa que es
deixaran al voltant de 20 entrades sense utilitzar. I no només això: aquest
microcontrolador disposa de molta més memòria de la que mai es necessitarà.(CITE)

El principal motiu per a fer el canvi és doncs el preu. Un \acro{AtMega32u4}
no és molt car (CITE), però és més barat un dispotitiu \acro{avr} de la sèrie
\acro{AtTinyXX}, com es veurà en el segïuent apartat.

\section{Segona versió de la placa}



\subsection{Digispark}
\subsection{Esquemàtic}

Finalment, s'observa que en el diagrama no hi ha cap element amb el que una
persona pugui interactuar directament amb el dispositiu (polsador, llum, ...).
Això és a propòsit, ja que es pretén que el dispositiu acabi tancat en una capsa
i que l'usuari no hagi de tocar-lo mai.

\subsection{Placa de circuit imprès}
\subsection{Producció de la \acro{pcb}} % JLCPCB
\section{3D Enclosure}

Quan es va saber que la placa impresa ja no rebria més modificacions físiques,
es va decidir crear una petita carcassa per a evitar que hi entri pols o les
ditades el puguin fer malbé. En un futur, aquesta carcassa també podrà protegir
al dispositiu d'altres agents externs, però en el moment del disseny l'objectiu
era tenir cura del dispositiu durant la fase de desenvolupament.

Així doncs, es va utilitzar el programa \est{FreeCAD} per a dissenyar una
carcassa amb dues peces: una superior i una inferior. Aquestes encaixarien entre
sí i, amb l'ajuda de cola quedarien subjectes. Les mesures s'han pres a partir
del model de \est{KiCad}, però també s'han corroborat amb la placa física i un
peu de rei.

A la figura \ref{todo} es poden veure les dues parts dissenyades. Com es pot
apreciar, només s'hi ha deixat un forat per a permetre el pas del connector
\acro{usb}. S'ha utilitzat parets de XX mm, i un marge entre les dues peces de
0.25 mm.

Es va aconseguir imprimir les dues peces amb una de les impressores de
l'escola, i es va aconseguir el resultat que es pot apreciar a la figura
\ref{xxx}. Finalment, es va comprovar que totes les peces encaixaven a la
perfecció i la placa cabia a dintre de la carcassa.

\section{OSHWA}

Un cop es va saber que el disseny físic del projecte era defintiu, es va decidir
publicar-lo a \acro{OSHWA}. L'\est{Open-Source HardWare Association} és, com el
nom indica, una associació que vetlla pel maquinari de codi obert. Una de les
diverses tasques que fa és mantenir un registre de dissenys de maquinari obert,
sempre que els autors ho autoritzin. CITE

Un dels beneficis de registrar el disseny en el seu sistema és que queda una
evidència (més) que el projecte s'ha fet, té la llicència que té, i protegeix
a l'autor contra possibles plagis. El més interessant de tot això és que aquest
servei s'ofereix gratuïtament: l'associació accepta donacions, però no son
necessàries per a poder registrar un disseny.

Així doncs, es va registrar el dispositiu a la seva base de dades i, després
de omplir el formulari i aquest ser revisat per l'associació, el dispositiu
va ser acceptat i es troba actualment sota la llicència \verb|ES000045|.
A la figura XX es troba el logo que es pot incloure al projecte, com a
resultat d'aquesta certificació.
