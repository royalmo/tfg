\chapter{Estudi econòmic i comercialització}
\label{cap:economia}

Aquest capítol té per objectiu simular el llançament del producte al mercat.
Hi ha diversos aspectes a tenir en compte, que es trobaran detallats en
diferents apartats.

\section{Preu del producte}

El primer aspecte important és el preu del producte. Per a poder definir un
preu de venda, primer s'ha de tenir present quin és el preu de cost. Durant
el transcurs del projecte s'ha produït una tirada de 5 unitats amb el
proveïdor \acro{jlcpcb}. A la taula \ref{tab:jlcpcb} hi ha el detall dels
costos de la tirada.

\begin{table}[ht]
    \centering
    \begin{tabular}{lc}
        \textbf{Concepte}           & \textbf{Cost} \\ \hline
        Impressió de la PCB         & 1,87€         \\
        Muntatge de la PCB          & 10,21€        \\
        Impressió 3D                & 28€           \\
        Cable USB2.0 A-C            & 5€            \\
        Enviament 15 dies           & 1,90€         \\
        Components (total)          & 62,47€        \\
        - C1: 4.7u (5u.)            & 0,65€         \\
        - C2,C4: 10u (10u.)         & 0,18€         \\
        - C3,C5,C6: 0.1u (15u.)     & 0,08€         \\
        - C7: 2200p (5u.)           & 0,17€         \\
        - R1,R2: 5.1k (10u.)        & 0,19€         \\
        - R3,R4: 68 (5u.)           & 0,20€         \\
        - R5: 1.5k (5u.)            & 0,01€         \\
        - R6,R7: 4.7k (10u.)        & 0,19€         \\
        - D1,D2: 3.3V (14u.)        & 0,52€         \\
        - J1: USBC (7u.)            & 0,60€         \\
        - U1: MCP1703T (5u.)        & 3,43€         \\
        - U2: AtTiny85 (5u.)        & 18,89€        \\
        - U3: MPU6050 (5u.)         & 37,36€        \\ \hline
        \textbf{TOTAL (IVA Inclòs)} & \textbf{109,45€}     
        \end{tabular}
    \caption{Cost de produir una tirada de 5 dispositius.}
    \label{tab:jlcpcb}
\end{table}

A la taula hi ha també els costos d'altres parts, com la impressió
3D i el connector \acro{usb}. En els dos casos s'ha aproximat el preu, ja que
o bé es disposava ja d'aquest material (i no se n'ha comprat), o bé l'escola
l'ha proporcionat gratuïtament.

Tal i com es pot veure, hi ha certs components on la quantitat més petita
d'unitats que es poden comprar és superior al que es necessitava. Tenint aquest
factor en ment i sabent que els preus solen disminuir quan s'augmenten les
quantitats, s'ha decidit simular la compra de 10000 unitats. A la taula
\ref{table:pricing10k} es pot consultar el desglossat de preus per a
grans quantitas.

%% TAULA 2
AFEGIR TAULA 2

Així doncs, si es produís aquest producte en massa, el preu de cost rondaria
els 10€. Tanmateix, el preu de venda sempre hauria de ser més elevat que
el preu de cost. El marge que es defineixi pot dependre de diversos factors.
A continuació es detallen alguns aspectes que s'haurien de tenir presents
a l'hora de fixar un preu:

\begin{itemize}
    \item Primer de tot s'hauria de realitzar un estudi de mercat, per a veure
    si hi ha productes similars, i el preu i èxit que tenen. En general es vol
    que el nou producte tingui preus similars als ja existents.
    \item Es pot complementar l'estudi anterior amb un estudi del \est{target},
    és a dir, preguntar als possibles compradors quin preu estan disposats a
    pagar per al producte.
    \item S'ha de valorar si es vol tenir en compte els costos de
    desenvolupament. Si aquest producte no s'hagués creat en el marc d'un TFG,
    s'hauria hagut de pagar a la persona que ho desenvolupés en funció del
    temps dedicat. Per a saber quant de percentatge dedicar al salari d'aquesta
    persona s'ha de fer una estimació del nombre d'unitats que es vendran.
    \item S'ha d'afegir al preu de cost les despeses que es puguin ocasionar
    durant la publicitat i venda del producte.
    \item S'ha de determinar, si s'escau, un marge de benefici pels inversors
    del projecte.
    \item Finalment, es pot arrodonir el preu final en funció a altres aspectes,
    d'entre els quals el psicològic: un producte a 19€ és més llaminer que un
    altre a 20€ (AFEGIR SOURCE).
\end{itemize}

Així doncs, tenint present les recomanacions anteriors, es podria valorar que
un bon preu de venda podria ser 19.80€. Aquest preu agafa per referència al
producte acabat completament, és a dir totes les imperfeccions i millores que
es citaran al calítol \ref{cap:future_work} haurien d'estar acabades.

\section{Comercialització}

Un cop es sap el preu del producte i s'està disposat a vendre'l, es pot
comenar a comercialitzar-lo. Tanmateix, resten alguns aspectes pendents per
definir, que es detallen en aquesta secció.

\subsection{Imatge}

Donar una imatge al producte, al projecte i a l'empresa és una part essencial
durant el procés de comercialització. Per aquest motiu s'ha de dsetinar una
partida del pressupost en tasques d'imatge.

La més important sol ser tenir una pàgina web on hi hagi tota la informació
necessaria de cara al client. És important que hi hagi enllaços per a comprar el
producte, veure valoracions d'altra gent, i contactar amb l'empresa.

Generar un logo o icona per a les aplicacions i altra documentació també és una
bona forma d'associar una imatge al producte. També és important que els clients
recordin el producte per la facilitat d'ús (mitjançant tutorials) i no per
els maldecaps que els hi pugui haver generat.

Tot i que no s'ha previst vendre en un futur proper el producte, s'ha creat una
petita icona per a l'aplicació i s'ha creat una pàgina web estàtica molt
senzilla. Ambdues coses es poden consultar visitant
\url{https://gyroscreen.ericroy.net} o el repositori del projecte.

A la placa de circuit imprès s'ha afegit l'enllaç a la pàgina web anterior.
S'ha de tenir present que, si el projecte creixés més, seria convenient tenir
una domini propi per al producte. Tanmateix, el fet de pertànyer a un domini
pare fa que heredi la imatge del mateix. Un clar exemple és la pàgina web de
la titulació que s'està cursant, \url{https://ocwitic.epsem.upc.edu}, que
hereda el nom de la UPC.

\subsection{Llicències del codi}

Un punt a acabar de debatre abans de llançar el producte és si es publicarà el
codi font. Tal i com s'ha vist en les condicions de la llibreria \acro{v-usb}, el
\est{firmware} s'hauria de penjar amb la llicència \acro{gpl-3}, excepte si es
vol pagar una llicència comercial, que en funció dels dispositius que es vulguin
produir s'hauria de pagar més o menys.

Per a simplificar les coses i evitar possibles problemes legals, s'ha decidit
publicar totes les parts del sistema sota la mateixa llicència que \acro{v-usb}:
\acro{gpl-3+}. Tanmateix, s'ha deixat clar a la documentació del projecte que
es pot contactar a l'autor del projecte per si es vol pactar una llicència menys
restrictiva.

\subsection{Llicència USB}

Tal i com s'ha comentat a l'apartat \ref{subsec:usb-if}, durant el desenvolupament
del projecte s'ha utilitzat codis \acro{vid} i \acro{pid} arbitraris, tenint
cura que no fóssin de productes coneguts. Ara bé, al treure el producte al
mercat és convenient reservar una parella de codis per a aquest dispositiu.

Al mateix apartat s'explica que hi ha entitats que, sense cobrar res, assignen
identificadors a projectes de codi i maquinari obert. Un d'aquests és
\est{pid.codes}, que utilitza el sistema co\l.laboratiu de \est{GitHub} per a
acceptar noves peticions.

Així doncs, s'ha so\l.licitat el \acro{vid/pid} 0x1209, 0xF4F4 per al projecte
\est{Gyroscreen}. En el moment de l'entrega del treball encara no hi ha hagut
resposta per part de la persona que manté el projecte. Cal tenir en compte que
la persona que gestiona \est{pid.codes} no guanya res a canvi, només hi perd
temps i diners (SOURCE pid.codes/faq), i per això no està sempre pendent de
les noves peticions.

\subsection{Inversió inicial de capital}

Finalment s'haurà de fer una inversió de diners per a poder produir els
dispositius. En funció del nombre de dispositius la inversió serà més important
o menys, però també els beneficis.

Trobar inversors no és cap tasca fàcil. Per això, si es necessita suport
econòmic, és molt important que s'acabi dedicant un temps prudencial a preparar
possibles visites o entrevistes amb persones que estiguin disposades a confiar
en el projecte.

També es poden utilitzar plataformes web co\l.laboratives, com per exemple
\est{kickstarter}, que agrupa petites aportacions de molts usuaris per a
aconseguir suficient capital per a un projecte gran.
