\chapter{Treball futur i millores}
\label{cap:future_work}

Tenint en compte les limitacions temporals que imposen aquests tipus de
treballs, sol ser comú que els resultats finals siguin parcialment incomplets.
En el cas del disseny d'un prototip, aquestes possibles millores es posen encara
més en evidència. Per aquest motiu es dedicarà el darrer capítol en anomenar
possibles millores al projecte.

La millora més important i urgent és acabar de perfilar el programari per als
sistemes operatius Windows i MacOS. En aquest treball només s'ha aconseguit
completar el progrmari per Linux, i per a garantir un dels objectius principals
del projecte, el sistema ha de funcionar en la gran majoria de Sistemes
Operatius. Es podria aprofitar per a verificar el funcionament del sistema en
altres distribucions de Linux, la més important sent ChromeOS, una distribució
cada cop més utilitzada.

Una altra millora que aportaria molt al projecte és la facilitat de
configuració del sistema. Actualment s'ha d'editar un fitxer, buscar els números
de sèrie dels dispositius i els identificadors de les pantalles. Totes aquestes
tasques es podrien englobar en una única interfície gràfica, compatible entre
els diversos sistemes operatius (usant, per exemple, aplicacions com
\est{flutter}). La intenció és que es segueixi podent utilitzar el sistema a
través de la línia de comandes, però fer-lo també accessible a persones sense
tants coneixements en la matèria.

La tercera tasca que es proposa és la comercialització del producte.
Com s'ha vist en el capítol \ref{cap:economia}, el cost de producció dels
prototips no ha estat gaire elevat, i una possible producció del producte en
quantitats elevades reduïria encara més el preu final del producte. Això el
faria més atractiu de cara a prossibles compradors. Així doncs, s'hauria de
millorar la imatge del producte i fer-ne una mica de publicitat. Un bon inici
podria ser la millora de la pàgina web, l'ús d'un domini únic pel projecte
i l'ús d'imatges i altre contingut gràfic propi.

Finalment, hi ha un seguit de propostes que suposarien una gran inversió de
temps, i a la vegada no aportarien una millora significativa a l'usuari. Un clar
exemple és la reescriptura del programari utilitzant llenguatges de baix nivell,
com \est{C} o \est{Rust}. Suposaria una millora en l'ús de recursos
computacionals però també implicaria dedicar-hi moltes hores.

El manteniment del projecte també s'hauria de tenir en compte com a treball
futur. Un clar exemple de tasca de manteniment és el canvi d'ús de \est{Xlib}
a \est{Wayland}. Tal i com s'ha comentat a l'apartat \ref{subsec:wayland},
aquest darrer projecte encara no està globalment implementat, i necessita un
parell d'anys per a acabar de desenvolupar-se. Passat aquest temps, s'haurà
de migrar el programa de Linux a aquest nou entorn.

Tal i com s'ha vist al llarg d'aquest treball, hi ha un interès personal en
continuar aquest projecte, pel que aquestes millores esmentades podrien ser
perfectament el llistat de tasques pendents per a completar el projecte.
Així doncs, dit d'una forma més poètica, aquest darrer paràgraf no acabarà en
punt i final, sinó en un punt i a part, per a reprendre-ho en un futur proper
i potser acabar-ho treient al mercat.
