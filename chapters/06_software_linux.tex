\chapter{Disseny del software a Linux}

La darrera peça del puzzle per a que el sistema pugui funcionar és el
programari del cantó de l'ordinador. Tal i com s'ha comentat al capítol
\ref{cap:estat-de-l-art}, el fet d'utilitzar \acro{iio} implica que el
programari que es disseny per Linux no serà compatible amb la resta de
sistemes operatius. També s'ha vist que \est{Xorg} només existeix en sistemes
basats en Linux.

Així doncs, aquest capítol té per objectiu dissenyar un programari senzill però
robust per a poder utilitzar el sistema en dispositius basats en Linux. Es
centraran les proves en la distribució de Linux Ubuntu, concretament en
les versions 18.04 i 24.04 \acro{lts} (\est{Long-Term Support}), però tot hauria
de funcionar de la mateixa forma amb la resta de distribucions.

\section{Requisits mínims del programari}

Un bon inici per a aquesta part del projecte és identificar què és el que
s'espera del programa: com s'ha d'insta\l.lar, com ha de funcionar i com
s'ha de configurar. D'aquesta forma, es veu els requisits del programa des del
punt de vista d'un usuari, fent sovint més senzill l'ús de l'aplicació.

Després de donar-hi algunes voltes s'ha decidit utilitzar Python per a
desenvolupar l'aplicació, degut a la gran disponibilitat de llibreries, el fet
que ja es troba insta\l.lat per defecte en moltes distribucions de Linux, i
la simplicitat de programació (en comparació als llenguatges de baix nivell).
S'ha decidit també que s'utilitzarà un servei de Linux per al programa principal.
Això implicarà alguna configuració addicional, però de ben segur que facilitarà
la vida als usuaris.

Un altre aspecte important del programa és la manera de configurar-lo. Tot i que,
idealment, el millor per als usuaris finals seria una interfície interactiva,
Linux sol funcionar amb fitxers per tot. Així doncs, s'ha decidit crear un fitxer
\fitx{.conf} per a guardar les configuracions de l'aplicació. Aquest fitxer té
un format pactat i descrit a la documentació del programa, i es podrà guardar en
diferents llocs específics del sistema, on el programa els buscarà. Crear una
interfície interactiva que modifiqui aquest fitxer no hauria de ser una tasca
gens complicada, i podria ser una perfecta ampliació al sistema.

Pel que fa a l'insta\l.lació, es dissenyarà un paquet de Debian (una de les
distribucions de Linux més populars) que insta\l.larà els requisits i configurarà
el que faci falta per al bon funcionament del programa, fent aquesta tasca
possible per a persones que no dominin tant la línia de comandes.

\section{Llibreries utilitzades}

Un cop es sap les tasques que s'haurà de fer és el moment de cercar si hi ha
alguna eina que pugui simplificar la tasca de desenvolupament. Tal i com s'ha dit
a l'apartat anterior, Python té moltes llibreries (o mòduls) creats per la
comunitat i disponibles per tothom. Coneixent les parts més difícils de
l'aplicació, es pot cercar si algú ja les ha implementat.

\subsection{\est{PyLibiio}}

El primer mòdul cercat és \est{PyLibiio}, que consisteix en la llibreria de
Linux \est{libiio} portada a Python \cite{Libiio}. La llibreria compilada acostuma
a estar insta\l.lada per defecte en la majoria de sistemes basats en Linux, però
la portabilitat a Python és un afegit que s'ha d'insta\l.lar.

Ja s'ha comentat en el capítol \ref{cap:estat-de-l-art} la importància
i beneficis d'utilitzar l'entorn de \acro{iio} per al programa d'aquest projecte,
així que si, a més a més, es pot fer des de l'alt nivell d'abstracció que ofereix
Python, encara millor.

Aquest mòdul no té molta complexitat, i menys si es segueix com a referència
algun dels exemples llistats. La majoria dels exemples consistien en replicar
amb Python les comandes d'exemple de la llibreria \est{libiio}.

Tanmateix, durant les proves de l'exemple de la comanda \ord|iio_info| s'ha
vist que aquesta no donava els mateixos resultats que el mateix programa implementat
en C. Quan no s'especifica cap argument en la línia de comandes, s'hauria 
d'utilitzar el primer grup de dispositius possible. En canvi, el programa d'exemple
s'aturava sense mostrar cap error.

Així doncs, amb la intenció de contribuir a millorar el projecte, es va
notificar de l'error i es va crear una \est{Pull Request}
\footnote{
    Una \est{Pull Request} és una forma de \est{GitHub} de notificar i debatre
    els canvis proposats en una branca d'un repositori. Si es creu convenient,
    es poden aplicar aquests canvis a la branca principal (acció de \est{merge})
    \cite{PullRequest}.
}, \cite{LibiioPR}. Aquesta encara està pendent d'aprovació (ja que sembla que
les persones encarregades de mantenir la llibreria no tenen molt de temps
disponible).

\subsection{\est{PyUdev}}

Durant el desenvolupament del programa es va veure la necessitat d'obtenir el
número de sèrie a partir de l'adreça d'un dispositiu \acro{iio}. Després de fer
recerca, es va veure que la forma més senzilla era utilitzant una comanda
de \acro{udev}. I, com no podia ser d'una altra manera, es va cercar si es podia
estalviar aquesta comanda i utilitzar una llibreria de Python en el seu lloc.

És important evitar utilitzar comandes ja que aquestes són més propenses a
variar i deixar de funcionar que la pròpia interfície de desenvolupament que
proporcionen les llibreries, com és el cas de \est{libudev}. Per aquest motiu,
quan es va coneixer l'existència de \est{PyUdev} es va fer el canvi en la
implementació \cite{Pyudev}.

Aquesta llibreria també disposava d'exemples, i no s'hi ha trobat cap
error aparent. Com també ha passat amb el mòdul anterior, la pròpia llibreria
ja ve insta\l.lada, però la portabilitat a Python s'ha de llistar com una
dependència del programa.

Quan encara no es sabia del tot segur si es faria servir \acro{iio} o, una capa
més per sota, el protocol \acro{hid}, també es va cercar si hi havia una
implementació per Python de la llibreria \est{libhid}. Es va trobar el mòdul
\est{CPython-HidApi}, un projecte molt complet amb, fins i tot, eines per a
depurar de forma interactiva dispositius \acro{hid} \cite{CpythonHid}.
Finalment es va descartar utilitzar
aquest afegit, ja que es va passar a utilitzar \acro{iio}.

\subsection{\est{PyRandr}}
\label{subsec:pyrandr}

Finalment, la tercera i última contribució externa de l'aplicació és el
mòdul \est{PyRandr}. A diferència de la resta, aquest mòdul no es troba
disponible al repositori oficial de Python \est{PyPi} i, per tant, no es pot
insta\l.lar amb la comanda \ord|pip3| o marcar com a dependència quan es crei
l'aplicació final.

Aquest projecte, disponible a \est{GitHub}, és una abstracció de la comanda
\ord|xrandr| \cite{Pyrandr}. S'ha vist en l'apartat \ref{subsec:xrandr} que la
llibreria
\est{Xlib} pot ser una mica complicada d'utilitzar, i aquesta comanda no ha
canviat gaire en bastant temps. Així doncs, es considera una bona alternativa
per a l'aplicació.

La persona que va desenvolupar \est{PyRandr} en un primer moment ha deixat de
mantenir el projecte. A més a més, aquest no va tenir en compte tots els casos
d'ús, deixant en el codi alguns errors. Ja hi ha gent que ha creat \est{forks}
(còpies del projecte) i \est{pull reuquests} per a proposar millores, però
l'autor se n'ha despreocupat.

Així doncs, s'ha realitzat també una còpia del projecte i s'ha fusionat diversos
canvis proposats per a la comunitat. També s'ha afegit alguna correcció pròpia,
com el canvi d'orientació per a pantalles principals \cite{PyrandrOwn}. Tanmateix, el canvi més
gran que s'ha fet per a que pugui tot funcionar per a aquest projecte es veurà
en l'apartat \ref{subsec:systemd_system}.

Com que no es pot llistar aquest programa com a dependència, el més sensat és
afegir el codi (d'unes 300 línies) al propi repositori del projecte. inicialment
es va implementar com a un submòdul de \est{git}, però al fer tantes
modificacions i al voler fusionar tot el programa en un fitxer es va acabar
copiant i enganxant el contingut necessari.

\section{Desenvolupament del programa}

Un cop conegudes totes les peces del programa només falta unir-les entre sí. En
aquest capítol es detallen només alguns aspectes rellevants en referència al codi
creat específicament per a aquest programa. Es recomana tenir a proximitat el
codi font del projecte per a poder entendre millor aquest apartat.

\subsection{Funcionalitats addicionals implementades}

Tot i que la funcionalitat principal del programa ja és coneguda, s'ha decidit
dur aquest programa una mica més enllà, afegint una mica de robustesa i
escalabilitat. Hi ha moltes petites millores, però en aquest apartat es vol
destacar-ne només tres de més rellevants:

\begin{itemize}
    \item El sistema s'ha dissenyat per a suportar diversos dispositius a la
    vegada. Basant-se en el número de sèrie de cada dispositiu i els codis
    identificadors de \ord|xrandr| per a les pantalles, es pot associar una
    orientació d'un dispositiu a una pantalla en concret. Així doncs, més
    d'una pantalla es pot beneficiar del projecte a la vegada.
    \item S'ha realitzat el concepte de mitjana mòbil o filtre de pas baix per
    a aplicar les rotacions: fins que no hi hagi $x$ lectures consecutives amb
    una orientació diferent no s'aplica el canvi. El valor de $x$ no
    s'especifica, ja que l'usuari el pot configurar al seu gust.
    \item S'ha aplicat el concepte d'\est{offset}. Pot ser que algunes persones
    insta\l.lin el dispositiu físic amb el cable mirant cap amunt, algunes cap
    avall, i no es pretén obligar-los a fer-ho d'una única manera. En canvi,
    s'ha deixat configurable l'orientació en la que es troba el sensor en
    referència a la pantalla, i cada cop que es llegeixi una lectura s'aplicarà
    un \est{offset} per a obtenir l'orientació real.
\end{itemize}

Aquestes millores, juntament amb els detalls mencionats en aquest capítol, han
convertit un programa que feia la seva tasca en un de robust, escalable i segur.

\subsection{Conversió de les mesures d'acceleració a orientació}

Durant tot el document s'ha parlat d'utilitzar un sensor d'acceleració quan el
que es vol obtenir és l'orientació del dispositiu. No és pas un error, més bé
un avançament al que s'explicarà en aquest apartat.

Tots els dispositius mòbils utilitzen un acceleròmetre per a esbrinar en quina
orientació està el dispositiu, tal i com s'ha avançat a l'apartat
\ref{sec:sensor_selection}. El que no s'ha dit encara és com es passa d'una
mesura a una altra.

La forma de fer-ho és ben senzilla: sabent que la gravetat de la terra sempre
apunta cap avall, i que l'acceleració quan el dispositiu està quiet és únicament
la de la gravetat, només cal saber cap a quina direcció s'està mesurant aquesta
acceleració per a detectar l'orientació del sensor. A \cite{PedleyTilt} es
detalla tota la
trigonometria que hi ha darrere d'aquesta explicació, que no s'afegeix en aquest
document ja que no és rellevant pel producte final.

Com s'ha dit al paràgraf anterior, no serà possible utilitzar aquest sistema en
llocs on no hi hagi gravetat, o el dispositiu estigui rebent acceleracions en
alguna altra direcció. Per exemple, durant l'enlairament d'un avió els
dispositius mòbils no responen bé als canvis d'orientació, però un cop
s'assoleix la velocitat de creuer (i per tant no hi ha acceleració) ja torna
a funcionar amb normalitat.

Si mai es volgués mesurar l'orientació d'una forma més robusta es pot combinar
l'acceleròmetre amb un sensor de velocitat angular. L'acceleròmetre ajuda a
calibrar el sensor, determinant sempre la direcció de la gravetat. Però un
cop ca\l.librat, qui dona més precisió és el sensor giroscopic o de velocitat
angular. Es pot consultar un exemple de sensor amb més precisió a
\cite{6702711}, però no
s'implementarà en aquest projecte degut a la complexitat afegida i els propòsits
del projecte (no es cerca precisió).

\subsection{Configuració del programa}

La forma principal (i gairebé única, tret d'alguns detalls) de configurar
l'aplicació és mitjançant un fitxer de configuració. Quan s'executi el programa
es cercarà en diferents llocs el fitxer de configuració. Si no es troba cap
fitxer el programa s'aturarà amb un codi d'error. Si el fitxer que es troba
conté un error, també s'aturarà el programa.

En el repositori del projecte hi ha un exemple de fitxer de configuració molt
ben documentat i detallat. Aquest fitxer es pot dividir en dues parts:

\begin{itemize}
    \item La part global, que conté paràmetres (tots amb valors per defecte) per
    a, per exemple, canviar el temps d'espera entre lectures del sensor, canviar
    el nombre de mostres per la mitjana mòbil o canviar el llindar en què es
    considera que el dispositiu està rotat o estirat. Totes aquestes opcions es
    poden obviar i s'utilitzarà els valors per defectes.
    \item En segon lloc hi ha un bloc específic per a cada grup sensor-pantalla.
    Cada bloc necessita l'identificador sèrie del sensor i el connector de la
    pantalla, així com l'orientació inicial del dispositiu en referència a la
    pantalla. Totes aquestes opcions son obligatòries.
\end{itemize}

Cada cop que l'usuari modifiqui el fitxer de configuracions haurà de reiniciar
el programa. Això significa que si una modificació temporal o a mitges es guarda
no afectarà al funcionament del programa fins que aquest es reinicii.

Un exemple de fitxer de configuració senzill podria ser el següent:

\begin{verbatim}
# Es poden deixar comentades les configuracions globals si es volen utilitzar
# els valors per defecte.
[Global]
#sample_interval_ms = 1000
#min_samples_to_change = 5
#sample_interval_when_change_detected = 300
#angle_threshold_low = 45
#angle_threshold_high = 75

# Configuració específica de la primera pantalla
[PantallaPrincipal]
# Número de sèrie del dispositiu
device = A001
# Identificador `xrandr` de la pantalla
display = HDMI-1
# A on apunta el connector en relació amb la pantalla.
# Pot ser 'Top', 'Bottom', 'Left' o 'Right'
connector_facing = Right

# Si es vol es pot definir més d'una pantalla
[PantallaExtra]
device = A002
display = eDP-0
connector_facing = Left
\end{verbatim}

Al repositori del projecte hi ha el mateix fitxer d'exemple però molt més ben
documentat. Aquí només s'ha posat els mínims comentaris per a entendre'l.

\subsection{Preparació del codi per a la distribució}

A diferència dels programes en C, que almenys els programes de característiques
similars a les d'aquest projecte es compilen en un únic fitxer executable, els
\est{scripts} de Python no s'interpreten en el mateix moment d'execució. Això
significa que, per a portar un programa d'un lloc a un altre s'ha de moure tot
el projecte, i no només l'executable.

Moure més d'un fitxer afegeix complexitat en el moment d'insta\l.lació i
configuració del projecte. Si aquest projecte fós molt gran s'assumirien les
conseqüències, però degut que si s'ajunten tots els fitxers quedaria un únic
fitxer de no més de 700 línies, s'ha considerat oportú agrupar tots els mòduls
en un únic fitxer.

Cal remarcar que els sistemes operatius basats en Linux ja tenen uns directoris
separats dels dels executables per a posar mòduls extres \cite{InstallPython3},
però es considera que
s'afegeix molta dificultat en quant a insta\l.lar el programa (sobretot en un
context d'usuari) i s'hi guanya poc.

\section{Execució del programa en segon pla: \est{systemd}}

Un cop el programa funcionava correctament amb la línia de comandes és el moment
de fer-lo funcionar en un segon pla. Tal i com s'ha detallat a l'apartat
\ref{subsec:systemd}, hi ha dos tipus de serveis que es poden crear, els dos
amb propòsits diferents. Per a oferir més possibilitats a l'usuari i, sobretot,
coneixer millor les diferències entre aquests dos tipus, s'ha decidit crear
un servei per usuari i un servei per al sistema.

\subsection{Servei d'usuari}

El procediment per a crear un servei de Linux a partir d'un programa de Python
no és un camí senzill, i s'ha optat per a seguir un tutorial molt ben cuidat i
detallat \cite{PythonSystemdTutorial}. Es recomana fortament utilitzar aquest
tutorial si es vol crear un servei per Python.

Tanmateix, hi ha un parell de coses que s'han acabat fent diferent en referència
al servei d'usuari. La intenció de crear un servei d'usuari és permetre que els
usuaris no administradors puguin utilitzar aquest programa encara que no estigui
insta\l.lat. Com que totes les dependències no insta\l.lades es poden baixar
sense permisos d'administrador amb la comanda \ord|pip3| això esdevé una tasca
molt senzilla.

Només s'ha canviat el lloc on estarà l'executable, a \fitx{~/.local/bin} i el
fitxer de configuracions, a \fitx{~/.config/gyroscreen}. La resta es mantindrà
tot igual. Els fitxers de servei utilitzen una notació específica per a
definir variables per al temps d'execució. Aquesta documentació es pot consultar
a \cite{mansystemd}.

També s'ha fet que el programa només informi a \est{systemd} que s'ha carregat
correctament el servei si aquest s'executa amb l'argument \ord|--daemon| a la
línia de comandes. Així es manté la funcionalitat del programa en entorns fora
de serveis.

\subsection{Servei de sistema}
\label{subsec:systemd_system}

Si es pretén insta\l.lar el programa per a tots els usuaris només cal seguir
els darrers passos del tutorial citat en l'apartat anterior. S'ha de moure de
lloc l'executable (a \fitx{/usr/bin}), i el servei (al lloc on indica el
tutorial). El fitxer de configuracions també es pot moure a \fitx{/etc} en funció
de si es vol que les configuracions siguin idèntiques per a tots els usuaris o
puguin canviar entre usuaris.

Tanmateix, la dificultat més gran per a utilitzar el programa com un servei de
sistema és que no hi ha les variables d'entorn de la sessió, i en conseqüència
\ord|xrandr| no funciona. S'ha explicat quines variables són i la importància
que tenen en l'apartat \ref{subsec:xrandr}.

Així doncs, toca modificar considerablement la part de les comandes del
projecte \est{PyRandr}, tal i com s'ha mencionat a l'apartat \ref{subsec:pyrandr}.
El procediment que s'implementarà per a buscar les variables d'entorn encertades
no és molt professional, però completament funcional: s'anirà provant per diferents
opcions de \verb|$DISPLAY| (concretament \verb|:0| i \verb|:1|) i es cercarà
en diferents directoris predefinits el fitxer \fitx{.Xauthority}, com pot ser
en els directoris de cada usuari o a \fitx{/run}.

\begin{figure}[ht]
    \centering
    \begin{adjustbox}{height=0.6\textheight}
    \begin{tikzpicture}[node distance=1.5cm]

        \node (start) [startstop] {Inici};
        \node (xr1) [process, below of=start] {Executar \texttt{xrandr}};
        \node (dec1) [decision, below of=xr1] {Funciona?};
        \node (ok1) [startstop, below of=dec1, xshift=-2cm] {Fi};
        \node (xauth) [process, below of=dec1, xshift=2cm] {Buscar \texttt{.Xauthority}};
        \node (found) [decision, below of=xauth] {Trobat?};
        \node (wait) [process, right of=found, xshift=2.5cm, yshift=0.8cm] {Esperar 30 segons};
        \node (xr2) [process, below of=found, yshift=-0.8cm] {
            Executar \texttt{xrandr} amb \texttt{.Xauthority} i \texttt{DEVICE=:0}
        };
        \node (dec2) [decision, below of=xr2, yshift=-0.8cm] {Funciona?};
        \node (ok2) [startstop, below of=dec2, yshift=-0.4cm, xshift=2cm] {Fi};
        \node (xr3) [process, below of=dec2, yshift=-0.6cm, xshift=-2cm] {
            Executar \texttt{xrandr} amb \texttt{.Xauthority}i \texttt{DEVICE=:1}
        };
        \node (dec3) [decision, below of=xr3, yshift=-0.8cm] {Funciona?};
        \node (ok3) [startstop, below of=dec3, yshift=-0.3cm] {Fi};
        
        \draw [arrow] (start) -- (xr1);
        \draw [arrow] (xr1) -- (dec1);
        \draw [arrow] (dec1.west) -| node[anchor=east] {Sí} (ok1.north);
        \draw [arrow] (dec1) -| node[anchor=west] {No} (xauth);
        \draw [arrow] (xauth) -- (found);
        \draw [arrow] (found) -| node[anchor=west] {No} (wait);
        \draw [arrow] (found) -- node[anchor=east] {Sí} (xr2);
        \draw [arrow] (wait) |- (xauth);
        \draw [arrow] (xr2) -- (dec2);
        \draw [arrow] (dec2) -| node[anchor=west] {Sí} (ok2);
        \draw [arrow] (dec2) -| node[anchor=east] {No} (xr3);
        \draw [arrow] (xr3) -- (dec3);
        \draw [arrow] (dec3) -- node[anchor=east] {Sí} (ok3);
        \draw [arrow] (dec3) -| node[anchor=west] {No} (wait);
        
    \end{tikzpicture}
    \end{adjustbox}

    \caption{Diagrama simplificat de decisió del context de \texttt{xrandr}.}
    \label{fig:xrandr-program-schema}
\end{figure}


A la Figura \ref{fig:xrandr-program-schema} es pot veure l'arbre de decisions
per a obtenir el context necessari per a \ord|xrandr|. El primer cop que es
vulgui executar l'ordre s'executarà el codi equivalent a aquest diagrama.
Un cop trobat el context necessari per a obtenir l'entorn de l'usuari es guardarà
en una variable per a no haver de repetir cada cop el procés. Si mai en algun
moment donat salta un error durant l'execució de \ord|xrandr| amb el context
guardat es tornarà a executar el codi anterior per a determinar de nou el
context necessari.

Com es pot intuir a la figura, és possible que aquest esquema de decisió no
s'acabi mai, és a dir, que es quedi en un bucle infinit. Això està fet a
propòsit: quan no es troba el context de cap manera sol ser perquè l'usuari ha
engegat l'ordinador però encara no ha iniciat la sessió. En aquests casos s'ha
decidit esperar mig minut i tornar a fer les comprovacions per veure si ja
s'ha inicialitzat un entorn gràfic. En el cas d'aquest programa no serviria de
res aturar-se amb un codi d'error, ja que al ser un servei el \est{kernel}
tornaria a executar el programa just quan s'aturés.

Així doncs, és possible que no es pugui recuperar el context de \est{xrandr}
després de tots aquests intents. Tanmateix, s'ha provat el
sistema amb diferents versions de Debian i Ubuntu i en totes ha acabat funcionant.
En un cas extrem, cal recordar que sempre es pot insta\l.lar l'aplicació per a
que funcioni en el context de la sessió, mitjançant un servei d'usuari.

\section{Insta\l.lació del programa}

Tal i com s'ha vist en l'apartat anterior, el servei de sistema necessita guardar
els fitxer de configuració i els fitxers executables en llocs molt específics.
Insta\l.lar el programa a mà, quan es tracta de fer-ho per a tot el sistema,
pot ser fins i tot una tasca perillosa. Per aquest motiu s'ha decidit crear un
paquet de Debian per a facilitar el procés d'insta\l.lació.

\subsection{Creació del paquet de Debian}

El més important durant la creació d'un paquet és saber quines dependències
necessita el programa i quina és la versió més baixa amb la que poden funcionar.
Per sort, s'ha desenvolupat el programa en un entorn de Ubuntu 18.04, que fa
pocs mesos que ha deixat de rebre suport gratuït \cite{UbuntuSupport}. Per tant,
aquest sistema
operatiu es considerarà com el més antic que suporta aquesta aplicació. En altres
paraules, les versions de les dependències insta\l.lades en aquest sistema seran
les que es posaran en el llistat de dependències del paquet.

A la Figura \ref{fig:software-dependency-graph} es pot veure el diagrama de
dependències de l'aplicació creada. Com es pot veure, hi ha dependències que
son directes o de primer nivell, però aquests programes amb els que depèn també
tenen al seu torn altres dependències. Tot plegat acaba amb un arbre de
dependències que, per a evitar fer molt gran l'esquema, s'ha retallat a només
les dependències més properes. Al paquet de Debian només s'hi haurà d'incloure
les de primer nivell (ja que aquestes ja inclouran al seu torn la resta).

\begin{figure}[ht]
    \centering
    \begin{adjustbox}{width=\linewidth}
    \begin{forest}
        forked edges,
        for tree={
            draw,
            align=center,
            edge={-latex}
        },
        boxd/.style={
            draw,
            dashed,
        },
        dotdep/.style={
            draw,
            dotted,
        },
        [\texttt{gyroscreen} v1.1, name=gyroscreen
            [\texttt{x11-xserver-utils}
                [\texttt{libx11}
                    [\dots, color=black!35, dotdep]
                ]
                [\texttt{libxrandr2}
                    [\dots, color=black!35, dotdep]
                ]
                [\dots, color=black!35, dotdep]
            ]
            [\texttt{python3}
                [\texttt{libpython3-stdlib}
                    [\dots, color=black!35, dotdep]
                ]
                [\dots, color=black!35, dotdep]
            ]
            [\texttt{python3-libiio}
                [\texttt{libiio0}
                    [\dots, color=black!35, dotdep]
                ]
                [\texttt{python3}, color=black!35, boxd]
            ]
            [\texttt{python3-pyudev}
                [\texttt{libudev1}
                    [\dots, color=black!35, dotdep]
                ]
                [\texttt{python3}, color=black!35, boxd]
            ]
        ]
    \end{forest}
    \end{adjustbox}

    \caption{Diagrama simplificat de dependències del paquet de Debian.}
    \label{fig:software-dependency-graph}
\end{figure}


Segurament que, després de consultar l'arbre de dependències de la Figura
\ref{fig:software-dependency-graph}, un es preguntarà perquè la dependència
de \texttt{python3} està de color gris en els dos paquets de Python. El motiu
d'aquest color rau en que aquesta dependència ja ha estat satisfeta durant el
primer nivell, pel que no s'ha de tornar a insta\l.lar. Tanmateix, convé seguir
tenint-la present, ja que si mai es desinsta\l.la el programa principal però es
vol mantenir \texttt{python3-libiio} no es podrà desinsta\l.lar \texttt{python3}.

També és essencial saber determinar per a quines arquitectures és compatible el
programa creat. Per exemple, si s'hagués implementat el projecte en C, s'hauria
de compilar per a cada arquitectura separadament, i crear un paquet per a cada
una d'elles. En el cas de Python, en canvi, és compatible per a totes les
plataformes. Per això es posarà \texttt{all} enlloc de una arquitectura específica,
com podria ser \texttt{amd64}.

La següent tasca més important és saber què s'ha d'insta\l.lar i a on. Per sort,
Debian facilita molt aquesta tasca, i per programes senzills gairebé que és
bufar i fer ampolles: només cal indicar a on es copia cada fitxer i l'empaquetador
fa la resta. Per exemple, si es copia un servei de sistema s'encarrega d'arrancar
el servei, si es copia un fitxer de configuració es mira que no s'estigui
sobreescrivint una versió anterior, o si es copia una pàgina de manual s'encarrega
de recarregar la base de dades dels manuals.

La resta de tasques, en canvi, són més aviat diplomàtiques: indicar la llicència
amb la que es distribueix el programa, l'autor, canvis que s'ha fet des d'una
versió anterior i organitzar-ho tot en carpetes, tal i com ho vol Debian.
Es recomana veure el tutorial de \cite{DebCreation} si es vol replicar aquests passos.

Finalment, s'ha creat un \est{shell script} per a poder crear fàcilment el
paquet. Tot i que el nombre de comandes a recordar son poques, és una tasca que
no es fa sovint, i es prefereix deixar-ho apuntat i ben preparat en algun lloc.
Es pot consultar en el repositori del treball aquest \est{script}.

\subsection{Creació d'una pàgina de manual}

Quan el paquet de Debian ja va funcionar correctament s'hagués pogut considerar
la part de programari per Linux completa. Tanmateix, mentre es creava el paquet,
l'eina de creació mostrava una alerta indicant que el programa no tenia cap
pàgina de manual. Això va portar al desig personal d'aprendre com es creaven
i com s'afegien en un paquet de Debian.

Per això es va cercar en diferents llocs fins que es va trobar l'eina
\ord|pandoc| \cite{PandocTutorial},
que converteix un fitxer en \est{Markdown} en una pàgina de manual amb el
format que aquesta necessita. El format de les pàgines de manual es considera
editable a mà, però la corba d'aprenentatge és prou alta i, segons moltes fonts,
no val la pena per a pàgines petites o projectes poc ambiciosos \cite{Manpage}.

Un cop creada la pàgina de manual, s'ha d'afegir en les comandes d'insta\l.lació
que es copiï al directori pertinent aquesta pàgina, i un cop insta\l.lat el
paquet de nou, només cal executar \ord|man gyroscreen| per a consultar el
manual d'ús del programa, en anglès.

\subsection{Possible ampliació: creació d'un repositori APT}

Insta\l.lar el paquet generat es pot fer utilitzant l'eina \ord|dpkg| sense cap
problema. L'únic inconvenient d'aquesta eina és que no insta\l.la automàticament
les dependències, però en el \acro{readme} del repositori es detallen tots els
passos per a seguir en aquests casos, que ve a ser la solució proposada a
\cite{dpkgHelp}.

Si es volgués distribuir més fàcilment el programa el pas a seguir seria
penjar el paquet en algun repositori \acro{apt}. Això implicaria tenir un servidor
web estàtic on poder descarregar els programes. Els usuaris haurien de, en primer
lloc, afegir la clau pública amb la que es signa el paquet al seu ordinador, i
llavors podrien afegir el repositori \acro{apt} en el seu llistat de repositoris
on buscar programes. Finalment, es podria descarregar i insta\l.lar el programa
utilitzant l'ordre \ord|apt| o \ord|apt-get|.

Tanmateix, el que sobre paper sembla molt senzill implica la configuració d'un
servidor, domini, i haver de signar amb una clau privada els paquets que es
generi. Per falta de temps s'ha decidit deixar aquesta tasca com a pendent, i
recuperar-la en un futur si el projecte té prou de suport.
