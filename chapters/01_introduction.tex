\chapter{Introducció}
\label{cap:introduccio}

Durant els darrers anys, el contingut digital en format vertical no ha cessat
d'augmentar, sent una de les principals causes l'ús generalitzat dels
dispositius mòbils \cite{Navarro2023El}. Les xarxes socials no van tardar a
adaptar les seves interfícies a aquestes noves resolucions. De fet, aquesta
tendència també s'ha portat a moltes altres aplicacions, tan de mòbil com
d'escriptori. Per exemple, han nascut termes com el
\est{Mobile-First Web Design}, que incita als desenvolupadors de pàgines web
a dissenyar les pàgines per a dispositius mòbils, i adaptar-les després
(el contrari del que es feia tradicionalment) \cite{varrela2015mobile}.

Tot plegat ha creat una nova moda: tenir una pantalla en vertical
en un ordinador de sobretaula. La majoria de monitors es poden muntar en
aquesta orientació sense necessitar un altre suport, pel que aquesta pràctica
s'ha popularitzat molt fàcilment \cite{WeardenPortrait}.
Disposar d'una pantalla en vertical també pot aportar altres avantatges. Per
exemple: els desenvolupadors poden veure més línies d'un mateix fitxer de codi
a la vegada, els dissenyadors de cartells o continguts per a dispositius mòbils
també poden beneficiar-se d'aquesta orientació.

Tanmateix, de vegades pot resultar més útil veure contingut en format
horizontal, com per exemple les pe\l.lícules. Si disposem de la configuració
anterior, hauríem de desmuntar la pantalla del suport i tornar-la a co\l.locar
en l'orientació desitjada. Un cop fet això, s'hauria d'anar a la configuració
del sistema per a que l'entorn gràfic del sistema operatiu mostri els continguts
d'aquella pantalla correctament. Aquest cúmul de tasques fan aquest procediment
tediós, sent aquest un motiu pel que no es sol dur a terme.

Per a resoldre aquest problema, han sortit a la venda suports rotatoris per a 
monitors \cite{DIGITUSUniversal}. Aquests redueixen el problema mencionat
anteriorment a només haver de girar la pantalla amb la força de la mà, i
configurar l'entorn gràfic. Algunes marques han anat més enllà, oferint
monitors incorporats amb un sensor de gravetat que, al detectar un canvi
d'orientació s'encarreguen d'actualitzar l'entorn gràfic \cite{LCLC}.

Malauradament, aquests últims tipus de productes tenen alguns punts negatius:
\begin{itemize}
    \item El fabricant només assegura que s'actualitzi l'orientació en l'entorn
    gràfic si el sistema operatiu és Windows.
    \item Aquest sistema d'actualització funciona per sobre del protocol de
    transmissió de l'imatge (en aquest cas, \acro{hdmi}), i no es garanteix el correcte
    funcionament en altres connexions.
    \item Finalment, s'ha de comprar una nova pantalla per a poder gaudir del
    sistema. És a dir, el fabricant no ven per separat un dispositiu que tingués
    el mateix propòsit i pogués allargar la vida útil d'una pantalla ja
    comprada.
\end{itemize}

Així doncs, l'objectiu principal d'aquest treball és crear un dispositiu que,
juntament amb un suport rotatori ja existent, pugui convertir la tasca de girar
el monitor en un gest habitual durant una rutina de treball. Es posarà èmfasi en
la facilitat d'insta\l.lació i ús, compatibilitat amb diferents sistemes
operatius, i a una hipotètica comercialització del producte.

Com és d'esperar, al mercat també hi ha sensors d'orientació i/o acceleròmetres
\acro{usb} \cite{Yocto3D}. Tanmateix, gairebé la totalitat d'ells utilitzen
\est{drivers} propis i no es basen en estàndards preestablerts, que acostumen
a reduir el temps de desenvolupament i millorar la robustesa del sistema. S'ha
decidit doncs que s'utilitzarà aquests dispositius com a referència per a
comparar els resultats que s'obtindran, però es crearà un dispositiu des de zero.

La motivació darrera l'elecció d'aquest treball rau en l'interès personal en
posar en pràctica totes les àrees de coneixement que componen la titulació que
s'està cursant. Addicionalment, també hi havia un interès personal per a obtenir
un producte final que pot aportar un ús en el dia a dia. Per aquest motiu, quan
es va presentar l'oportunitat de dissenyar el producte mencionat anteriorment,
es va pensar que era una molt bona manera per a donar un final rodó al grau.

Aquesta documentació s'haurà de complementar amb el codi font del projecte, que
segurament es trobarà juntament amb aquest fitxer. En cas que no es disposi o
es vulgui consultar la darrera versió, es trobarà el codi font a
\url{https://github.com/royalmo/gyroscreen}.
Finalment, es vol mencionar que aquest document s'ha generat amb l'eina \LaTeX,
i es poden consultar els fitxers font a \url{https://github.com/royalmo/tfg}.
