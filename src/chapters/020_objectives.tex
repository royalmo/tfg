\chapter{Objectius}
\label{cap:objectius}

\section{Objectius generals}

L'objectiu principal d'aquest treball de fi de grau és implementar un dispositiu
de dimensions reduïdes, que s'enganxarà darrere el monitor i es connectarà
a l'ordinador mitjançant un cable \acro{usb}. Aquest dispositiu contindrà els
sensors i altres components necessaris per poder informar l'ordinador sobre
l'orientació actual de la pantalla. A través d'un programari propi i de la
interfície de programació que proporciona l'entorn gràfic, es canviarà
l'orientació de la pantalla en funció de les dades rebudes del dispositiu. Es
buscarà sempre la màxima compatibilitat possible, facilitat d'insta\l.lació i
ús quotidià, i un preu de construcció raonable.


\section{Objectius específics}

Aquest treball de fi de grau es podria extendre molt més enllà dels pocs mesos
de temps dels que es disposa. Per això, es definiran uns objectius essencials
per considerar el projecte completat, i uns objectius addicionals o
ampliacions que, tot i no ser troncals pel projete, poden completar o
millorar el sistema.

Objectius essencials:
\begin{itemize}
    \item Fer una recerca sobre la tecnologia i pràctiques actuals sobre el 
    procés de disseny de dispositius \acro{usb}.
    \item Escollir els components electrònics del dispositiu i dissenyar una
    placa de circuit imprès. S'imprimiran unes quantes plaques per poder fer
    proves del sistema complet.
    \item Definir la forma en què l'usuari podrà insta\l.lar i configurar
    el sistema.
    \item Adaptar una implementació o implementar el programari que comunicarà
    el dispositiu i l'ordinador. Aquest aplicatiu haurà de complir els
    requisits d'interració de l'usuari definits en el punt anterior. En aquest
    punt es centrarà només en l'entorn de GNU/Linux.
\end{itemize}

Objectius addicionals:
\begin{itemize}
    \item Adaptar el sistema a Windows i MacOS fent-lo, doncs, compatible amb la
    gran majoria de sistemes operatius del mercat.
    \item Un cop es tingui la placa, dissenyar i construir una carcassa pel
    dispositiu.
    \item Dissenyar i implementar una interfície d'usuari senzilla i amable
    que permeti configurar el dispositiu.
    \item Realitzar un estudi econòmic per a una possible comercialització del
    projecte.
    \item Dissenyar una pàgina web senzilla per promocionar el projecte.
    \item Preparar el sistema per poder acceptar més d'un dispositiu
    simultàniament.
\end{itemize}
