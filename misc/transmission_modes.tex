\begin{figure}[ht]
    \centering
    \begin{tabular}{l p{0.6\linewidth}}
        \toprule
        \textbf{Mode}           & \textbf{Descripció} \\
        \midrule
        Control & Mode utilitzat durant la configuració del dispositiu per enviar comandes específiques que requereixen el protocol. \\
        Isòcron & Mode utilitzat per transmetre dades a temps real (audio, vídeo, \dots). Aquest mode reserva una porció de l'amplada de banda \acro{usb} per garantir la mateixa latència. No es tornen a enviar paquets amb errors. \\
        Per interrupció & Aquest mode demana als dispositius cada cert temps si han de transmetre dades. Utilitzat per a dispositius que requereixen transmetre poques dades, com teclats o ratolins. \\
        Massiu (\est{Bulk}) & Mode utilitzat per a dispositius que necessiten transmetre moltes dades i garantir l'absència d'errors, però que no tenen requisits de latència. Ho són, per exemple, les impressores i els discs durs. \\
        \bottomrule
    \end{tabular}
    \caption{Modes de transmissió \acro{usb} \cite{Axelson2015USB}.}
    \label{tab:transmision-modes}
\end{figure}
